\documentclass[11pt]{article}

\usepackage{graphicx}
\usepackage{amsthm}
\usepackage{dsfont}
\usepackage{amssymb}
\usepackage{amsmath}
\usepackage{listings}
\usepackage{fancyhdr}
\pagestyle{fancy}
\usepackage{indentfirst}
\usepackage{layout}
\usepackage{hanging}
\usepackage{setspace}
\usepackage{mathtools}
\usepackage{physics}

\DeclarePairedDelimiter\qb{\lvert}{\rangle}

\def\tit{Quantum Programming HW6}
\def\term{February 2020}

\graphicspath{{.}}

\def\auths{Joshua Larkin}

\doublespacing

\lhead{\term}
\chead{\tit}
\rhead{\thepage}
\cfoot{}


\title{
    \vspace{2in}
    \textmd{\textbf{\tit}}\\
    \normalsize\vspace{0.1in}\small{B490 : Spring 2020 }\\
    \vspace{0.1in}\large{\textit{\auths}}
    \vspace{3in}
}

\date{}

\newcommand{\icol}[1]{
  \left(\begin{smallmatrix}#1\end{smallmatrix}\right)
}

\def\bfi{\ensuremath\textbf{i}}
\def\srtt{\ensuremath\frac{1}{\sqrt{2}}}
\def\em{\ensuremath e^{\frac{-\textbf{i}\pi}{4}}}
\def\ep{\ensuremath e^{\frac{\textbf{i}\pi}{4}}}
\newcommand{\kzm}[1]{\begin{bmatrix}#1 \\ 0 \end{bmatrix}}
\newcommand{\kom}[1]{\begin{bmatrix}0 \\ #1 \end{bmatrix}}

\newcommand{\re}[1]{$\text{Re }$#1}
\newcommand{\im}[1]{$\texttt{Im }#1$}

\renewcommand\headrulewidth{0.4pt}
\fancyheadoffset{0.5 cm}

\oddsidemargin 0pt
\evensidemargin 0pt
\topmargin -.3in
\headsep 20pt
%\footskip 20pt
\textheight 8.5in
\textwidth 6.25in

\setlength\topmargin{0pt}
\addtolength\topmargin{-\headheight}
\addtolength\topmargin{-\headsep}
\setlength\oddsidemargin{0pt}
\setlength\textwidth{\paperwidth}
\addtolength\textwidth{-2in}
\setlength\textheight{\paperheight}
\addtolength\textheight{-2in}

\begin{document}

%%%% TITLE PAGE
\maketitle
\pagebreak

%%%%% First content Page
% \section{Work}
% Do exercises 1.1.2 thru 1.3.12 and exercises 2.1.1 thru 2.2.7
% Programming drills are optional

%%%%%%%%%%
%%%%%%%%%%%%%%%%

\section{Write the DFT matrix for $N = 2,3,4, \text{ and } 5.$}
\begin{itemize}
	\item[$N = 2$:] $$\omega = e^{i \pi} = -1$$
	\begin{align*}
		\frac{1}{\sqrt{2}} \begin{bmatrix}
		1 & 1 \\
		1 & \omega \\
	\end{bmatrix} &= 
	\frac{1}{\sqrt{2}}
	\begin{bmatrix}
		1 & 1 \\
		1 & -1 \\
	\end{bmatrix}
	\end{align*}

%%%%%%%%%%%%%%%%%%%%%%%%%%%%%%%%%%%%%

\item[$N = 3$:] $$\omega = e^{\frac{2 i \pi}{3}}$$
	\begin{align*}
	\frac{1}{\sqrt{3}} 
		\begin{bmatrix}
		1 & 1 & 1 \\
		1 & \omega & \omega^2 \\
		1 & \omega^2 & \omega^4 \\
		\end{bmatrix} &=
		\frac{1}{\sqrt{3}}
		\begin{bmatrix}
		1 & 1 & 1 \\
		1 & \omega & \omega^2 \\
		1 & \omega^2 & \omega \\
		\end{bmatrix} =
		\frac{1}{\sqrt{3}}
		\begin{bmatrix}
		1 & 1 & 1 \\
		1 & e^{\frac{2 i \pi}{3}} & e^{\frac{4 i \pi}{3}} \\
		1 & e^{\frac{4 i \pi}{3}} & e^{\frac{2 i \pi}{3}} \\
		\end{bmatrix}
	\end{align*}

%%%%%%%%%%%%%%%%%%%%%%%%%%%%%%%%%%%%%

\item[$N = 4$:] 
	\begin{align*}
		\omega &= e^{\frac{2 i \pi}{4}} = e^{\frac{i \pi}{2}} = i\\
		\omega^2 &= e^{i \pi} = -1 \\
		\omega^3 &= e^{\frac{3i\pi}{2}} = -i \\
		\omega^4 &= e^{2i\pi} = 1 \\
		\omega^6 &= e^{\frac{6 i \pi}{2}} = e^{3i\pi} = e^{i\pi} = \omega^2 \\
		\omega^9 &= e^{\frac{9 i \pi}{2}} = e^{\frac{i\pi}{2}} = i = \omega 
	\end{align*}
	\begin{align*}
	\frac{1}{\sqrt{4}}
	\begin{bmatrix}
		1 & 1 & 1 & 1 \\
		1 & \omega & \omega^2  & \omega^3 \\
		1 & \omega^2 & \omega^4 & \omega^6 \\
		1 & \omega^3 & \omega^6 & \omega^9 \\
	\end{bmatrix} &=
	\frac{1}{\sqrt{4}}
	\begin{bmatrix}
		1 & 1 & 1 & 1 \\
		1 & \omega & \omega^2  & \omega^3 \\
		1 & \omega^2 & \omega^4 & \omega^2 \\
		1 & \omega^3 & \omega^2 & \omega \\
	\end{bmatrix} =
	\frac{1}{\sqrt{4}}
	\begin{bmatrix}
		1 & 1 & 1 & 1 \\
		1 & i & -1  & -i \\
		1 & -1 & 1 & -1 \\
		1 & -i & -1 & i \\
	\end{bmatrix}
	\end{align*}

%%%%%%%%%%%%%%%%%%%%%%%%%%%%%%%%%%%%%

\item[$N = 5$] $$\omega = e^{\frac{2 i \pi}{5}}$$
	\begin{align*}
	\frac{1}{\sqrt{5}}
	\begin{bmatrix}
		1 & 1 & 1 & 1 & 1 \\
		1 & \omega & \omega^2 & \omega^3 & \omega^4 \\
		1 & \omega^2 & \omega^4 & \omega^6 & \omega^8 \\
		1 & \omega^3 & \omega^6 & \omega^9 & \omega^{12} \\
		1 & \omega^4 & \omega^8 & \omega^{12} & \omega^{16}\\
	\end{bmatrix} &=
	\frac{1}{\sqrt{5}}
	\begin{bmatrix}
		1 & 1 & 1 & 1 & 1 \\
		1 & e^{\frac{2 i \pi}{5}} & e^{\frac{4 i \pi}{5}} & e^{\frac{6 i \pi}{5}} & e^{\frac{8 i \pi}{5}}\\
		1 & e^{\frac{4 i \pi}{5}} & e^{\frac{8 i \pi}{5}} & e^{\frac{12 i \pi}{5}} 
		& e^{\frac{16 i \pi}{5}} \\
		1 & e^{\frac{6 i \pi}{5}} & e^{\frac{12 i \pi}{5}} & e^{\frac{18 i \pi}{5}} & e^{\frac{24 i \pi}{5}} \\
		1 & e^{\frac{8 i \pi}{5}} & e^{\frac{16 i \pi}{5}} & e^{\frac{24 i \pi}{5}} & e^{\frac{32 i \pi}{5}}\\
	\end{bmatrix}
	\end{align*}


%%%%%%%%%%%%%%%%%%%%%%%%%%%%%%%%%%%%%


\end{itemize}


\section{Verify that all four matrices above are unitary.}
A unitary matrix is defined to be a complex-valued square matrix whose inverse is equal to 
its conjugate transpose. 
That is, if $M$ is unitary, then $M^{-1} = \overline{M}^T$ such that $M \times M^{-1} = 1$ where 1 is the identity matrix. 

It is worth noting at first that each matrix $M$ is its own transpose. So, we just need to consider the conjugate of each matrix, multiply the original with the conjugate and check that the result is the identity matrix.

In our verifaction, the matrix on the left is the original matrix $M$ and the matrix on the right is $\overline{M}^T$. 

\begin{itemize}
	\item[$N = 2$:] 
	\begin{align*}
	\frac{1}{\sqrt{2}}
	\begin{bmatrix}
		1 & 1 \\
		1 & -1 \\
	\end{bmatrix}
	\times
	\frac{1}{\sqrt{2}}
	\begin{bmatrix}
		1 & 1 \\
		1 & -1 \\
	\end{bmatrix} &= 
	\frac{1}{2}
	\begin{bmatrix}
		2 & 0 \\
		0 & 2 \\
	\end{bmatrix} = 
	\begin{bmatrix}
		1 & 0 \\
		0 & 1 \\
	\end{bmatrix}
	\end{align*}

%%%%%%%%%%%%%%%%%%%%%%%%%%%%%%%%%%%%%
\newpage
\item[$N = 3$:] 
	$$\frac{1}{\sqrt{3}}
	\begin{bmatrix}
	1 & 1 & 1 \\
	1 & e^{\frac{2 i \pi}{3}} & e^{\frac{4 i \pi}{3}} \\
	1 & e^{\frac{4 i \pi}{3}} & e^{\frac{2 i \pi}{3}} \\
	\end{bmatrix} \times 
	\frac{1}{\sqrt{3}}
	\begin{bmatrix}
	1 & 1 & 1 \\
	1 & e^{\frac{-2 i \pi}{3}} & e^{\frac{-4 i \pi}{3}} \\
	1 & e^{\frac{-4 i \pi}{3}} & e^{\frac{-2 i \pi}{3}} \\
	\end{bmatrix}$$
	$$\frac{1}{3} 
	\begin{bmatrix}
		3 & (1 + e^{\frac{-2 i \pi}{3}} + e^{\frac{-4 i \pi}{3}}) &  (1 + e^{\frac{-4 i \pi}{3}} + e^{\frac{-2 i \pi}{3}}) \\ 
		(1 + e^{\frac{2 i \pi}{3}} + e^{\frac{4 i \pi}{3}}) &
		(1 + e^{\frac{2 i \pi}{3}}e^{\frac{-2 i \pi}{3}} + e^{\frac{4 i \pi}{3}}e^{\frac{-4 i \pi}{3}}) & 
		(1 + e^{\frac{2 i \pi}{3}}e^{\frac{-4 i \pi}{3}} + e^{\frac{4 i \pi}{3}}e^{\frac{-2 i \pi}{3}}) \\
		(1 + e^{\frac{4 i \pi}{3}} + e^{\frac{2 i \pi}{3}}) & 
		(1 + e^{\frac{4 i \pi}{3}}e^{\frac{-2 i \pi}{3}} + e^{\frac{2 i \pi}{3}}e^{\frac{-4 i \pi}{3}}) & 
		(1 + e^{\frac{4 i \pi}{3}}e^{\frac{-4 i \pi}{3}} + e^{\frac{2 i \pi}{3}}e^{\frac{-2 i \pi}{3}}) \\
	\end{bmatrix}$$
	 $$\frac{1}{3} 
	\begin{bmatrix}
		3 & (1 + -1) &  (1 + -1) \\ 
		(1 + -1) & (1 + 1 + 1) & (1 + -1) \\
		(1 + -1) & (1 + -1) & (1 + 1 + 1) \\
	\end{bmatrix}$$
	$$
	\begin{bmatrix}
		1 & 0 & 0 \\ 
		0 & 1 & 0 \\
		0 & 0 & 1 \\
	\end{bmatrix}$$

%%%%%%%%%%%%%%%%%%%%%%%%%%%%%%%%%%%%%
\newpage
\item[$N = 4$:] 
	$$\frac{1}{\sqrt{4}}
	\begin{bmatrix}
		1 & 1 & 1 & 1 \\
		1 & i & -1  & -i \\
		1 & -1 & 1 & -1 \\
		1 & -i & -1 & i \\
	\end{bmatrix} 
	\times
	\frac{1}{\sqrt{4}}
	\begin{bmatrix}
		1 & 1 & 1 & 1 \\
		1 & -i & -1  & i \\
		1 & -1 & 1 & -1 \\
		1 & i & -1 & -i \\
	\end{bmatrix} 
	$$
	$$\frac{1}{4}(
	\begin{bmatrix}
		1 & 1 & 1 & 1 \\
		1 & i & -1  & -i \\
		1 & -1 & 1 & -1 \\
		1 & -i & -1 & i \\
	\end{bmatrix} 
	\times
	\begin{bmatrix}
		1 & 1 & 1 & 1 \\
		1 & -i & -1  & i \\
		1 & -1 & 1 & -1 \\
		1 & i & -1 & -i \\
	\end{bmatrix})
	$$
	$$
	\frac{1}{4}(
	\begin{bmatrix}
	(1 + 1 + 1 + 1) & (1 + i + -1 + -i) & (1 + -1 + 1 + -1) & (1 + i + -1 + -i) \\
	(1 + i + -1 + -i) & (1 + i \times -i + 1 + -i \times i) & (1 + -i + -1 + i) & (1 + i \times i + 1 + -i \times -i) \\
	(1 + -1 + 1 + -1) & (1 + i + -1 + -i) & (1 + 1 + 1 + 1) & (1 + -i + -1 + i) \\
	(1 + -i + -1 + i) & (1 + -i \times -i + 1 + i \times i) & (1 + i + -1 + -i) & (1 + i \times -i + 1 + i \times -i) \\
	\end{bmatrix}
	)
	$$
	$$
	\frac{1}{4}(
	\begin{bmatrix}
	4 & 0 & 0 & 0 \\
	0 & 4 & 0 & 0 \\
	0 & 0 & 4 & 0 \\
	0 & 0 & 0 & 4 \\
	\end{bmatrix}
	)
	$$
	$$
	\begin{bmatrix}
	1 & 0 & 0 & 0 \\
	0 & 1 & 0 & 0 \\
	0 & 0 & 1 & 0 \\
	0 & 0 & 0 & 1 \\
	\end{bmatrix}
	$$
%%%%%%%%%%%%%%%%%%%%%%%%%%%%%%%%%%%%%
\newpage
\item[$N = 5$.]  
	Let $\omega = e^{\frac{i\pi}{5}}$
	$$
	\frac{1}{\sqrt{5}}
	\begin{bmatrix}
		1 & 1 & 1 & 1 & 1 \\
		1 & \omega^2 & \omega^4    & \omega^6     & \omega^8     \\
		1 & \omega^4 & \omega^8    & \omega^{12}  & \omega^{16}  \\
		1 & \omega^6 & \omega^{12} & \omega^{18}  & \omega^{24}  \\
		1 & \omega^8 & \omega^{16} & \omega^{24}  & \omega^{32}  \\
	\end{bmatrix}
	\times
	\frac{1}{\sqrt{5}}
	\begin{bmatrix}
		1 & 1 & 1 & 1 & 1 \\
		1 & \omega^{-2} & \omega^{-4}  & \omega^{-6}   & \omega^{-8}   \\
		1 & \omega^{-4} & \omega^{-8}  & \omega^{-12}  & \omega^{-16}  \\
		1 & \omega^{-6} & \omega^{-12} & \omega^{-18}  & \omega^{-24}  \\
		1 & \omega^{-8} & \omega^{-16} & \omega^{-24}  & \omega^{-32}  \\
	\end{bmatrix}
	$$
	% We claim the following: 
	% $1 + e^{\frac{-2 i \pi}{5}} + e^{\frac{-4 i \pi}{5}} + e^{\frac{-6 i \pi}{5}} + e^{\frac{-8 i \pi}{5}} = 0$
	The top row will be the following 5 expressions:
	\begin{align*}
	(1 + 1 + 1 + 1 + 1) \\
	(1 + \omega^{-2} + \omega^{-4} + \omega^{-6} + \omega^{-8}) \\
	(1 + \omega^{-4} + \omega^{-8} + \omega^{-12} + \omega^{-16}) \\
	(1 + \omega^{-6} + \omega^{-12} + \omega^{-18} + \omega^{-24}) \\
	(1 + \omega^{-8} + \omega^{-16} + \omega^{-24} + \omega^{-32}) 
	\end{align*}
	Take $k$ such that $k + 8$ is a multiple of 5. We make the following observation (using the fact that we are in an integral domain):
	\begin{align*}
	\omega^k(1 + \omega^{k} + \omega^{k+2} + \omega^{k+4} + \omega^{k+6}) 
		&= (\omega^k + \omega^{k+2} + \omega^{k+4} + \omega^{k+6} + \omega^{k+8}) \\
		&= (\omega^k + \omega^{k+2} + \omega^{k+4} + \omega^{k+6} + 1) \\
		&\Rightarrow (1 + \omega^{k} + \omega^{k+2} + \omega^{k+4} + \omega^{k+6}) = 0
	\end{align*}
	So then our top row is really just $5, 0, 0, 0, 0$.
	The same is true for the left column, because the only difference is the negation of the power, which doesn't actually affect our result.
	Note that our diagonal values will be as follows
	\begin{align*}
	1 + \omega^k\omega^{-k} + \omega^{k+2}\omega^{-(k+2)} + \omega^{k+4}\omega^{-(k+4)} + \omega^{k+6}\omega^{-(k+6)} 
		&= 1 + \omega^0 + \omega^0 + \omega^0 + \omega^0 \\
		&= 1 + 1 + 1 + 1 + 1 \\
		&= 5
	\end{align*}
	The rest of the matrix follows the same patterns as we have outlined above, so all non-diagonal values are 0. 
	$$
	\frac{1}{5}
	\begin{bmatrix}
		5 & 0 & 0 & 0 & 0 \\
		0 & 5 & 0 & 0 & 0 \\
		0 & 0 & 5 & 0 & 0 \\
		0 & 0 & 0 & 5 & 0 \\
		0 & 0 & 0 & 0 & 5 \\
	\end{bmatrix} 
	= 
	\begin{bmatrix}
		1 & 0 & 0 & 0 & 0 \\
		0 & 1 & 0 & 0 & 0 \\
		0 & 0 & 1 & 0 & 0 \\
		0 & 0 & 0 & 1 & 0 \\
		0 & 0 & 0 & 0 & 1 \\
	\end{bmatrix}
	$$
\end{itemize}


\section{Implement the $O(N^2)$ DFT algorithm in Python}
Submitted as $\texttt{dft.py}$

\section{Quantum Circuit implementation of DFT}
% Describe it and argue that the implementation has compexlity $O(\log^2N)$
The Quantum Fourier Transform can be thought of as a recursive algorithm. 
A single qubit has 2 states, i.e. $N = 2 = 2^1$. The circuit in this case is the 
Hadamard gate. For $n$ bits, we have $N = 2^n$ states. We construct the matrix by
recurring on half of the bits twice, at the same time. We also need to 
add rotation gates between the values produced by recursion, so we have a series
of rotations controlled by the previous $n-1$ qubits.
In matrix form this looks like (thanks to the QUIC Seminar paper given as resource):
$$\textbf{QFT}_N = \begin{bmatrix}\textbf{QFT}_{\frac{N}{2}} & \textbf{R}_{\frac{N}{2}} & \textbf{QFT}_{\frac{N}{2}} \\
	\textbf{QFT}_{\frac{N}{2}} & \textbf{R}_{\frac{N}{2}} & 
	\textbf{QFT}_{\frac{N}{2}} \end{bmatrix}$$
The two recursions on half as many arguments leads us towards the $O(\log^2 N)$ complexity. Another way to see this is that each wire gets a Hadamard gates, giving us $n$ gates.
Then the wires, starting with the most significant, get $n-1$ controlled rotations, 
then $n-2$ controlled rotations, continuing in this manner until the least significant 
which gets none (remember it only gets a Hadamard!).
Then our amount of gates is $n + (n-1) + \dots + 1 = \frac{n(n+1)}{2}$, which shows 
our $\log^2 N$ complexity.
\newpage
\section{Useful Definitions}
$$\ket{\textbf{i}} := \frac{1}{\sqrt{2}}(\ket{0} + i \ket{1})$$
$$\ket{-\textbf{i}} := \frac{1}{\sqrt{2}}(\ket{0} - i \ket{1})$$
Definition: Let $H$ be a two-dimensional complex Hilbert space. Let $h_1, h_2$ be unit vectors in 
$H$. 

$h_1$ and $h_2$ are equivalent as quantum states if there exists a modulus one complex number $c$ 
such that $h_1  = ch_2$.


\section{Rieffel and Polak 2.2}
Which pairs of expressions for quantum states represent the same state?

\begin{enumerate}
\item[$\textbf{a.}$] $\ket{0}$ and $-\ket{0}$ \\
	Take $c = -1$. Then $c\ket{0} = -\ket{0}$
\item[$\textbf{b.}$] $\ket{1}$ and $\textbf{i}\ket{1}$ \\
	Take $c = \textbf{i}$. Then $c\ket{1} = \textbf{i}\ket{1}$
\item[$\textbf{c.}$] 
	$\frac{1}{\sqrt{2}}(\ket{0} + \ket{1})$ and 
	$\frac{1}{\sqrt{2}}(-\ket{0} + \textbf{i}\ket{1})$ \\
	Note that 
	$$\frac{1}{\sqrt{2}}(\ket{0} + \ket{1}) = \begin{bmatrix}\frac{1}{\sqrt{2}} \\ 
			  		\frac{1}{\sqrt{2}}\end{bmatrix}$$
	$$\frac{1}{\sqrt{2}}(-\ket{0} + \textbf{i}\ket{1}) = 
		\begin{bmatrix}-\frac{1}{\sqrt{2}} \\ 
				 \frac{\textbf{i}}{\sqrt{2}}\end{bmatrix}$$

	Suppose towards a contradiction that there is such a number $c = a + ib$. \\
	So we must have $$c \cdot \begin{bmatrix}\frac{1}{\sqrt{2}} \\ 
			  		\frac{1}{\sqrt{2}}\end{bmatrix} 
					= \begin{bmatrix}-\frac{1}{\sqrt{2}} \\ 
				 \frac{\textbf{i}}{\sqrt{2}}\end{bmatrix}$$
		Then $$(a + ib) \times (\frac{1}{\sqrt{2}} + i0) = (-\frac{1}{\sqrt{2}} + i0)$$ 
		and \\ $$(a + ib) \times (\frac{1}{\sqrt{2}} + i0) = (\frac{i}{\sqrt{2}} + i0)$$ 
		But this is not possible, so there is a contradiction. 
		Hence there is no such $c$ and these two expressions represent different quantum states.
	
\item[$\textbf{d.}$] 
	$\frac{1}{\sqrt{2}}(\ket{0} + \ket{1})$ and $\frac{1}{\sqrt{2}}(\ket{0} - \ket{1})$\\
	$$\frac{1}{\sqrt{2}}(\ket{0} + \ket{1}) = \begin{bmatrix}
			\frac{1}{\sqrt{2}} \\ 
			\frac{1}{\sqrt{2}}\end{bmatrix}$$
	$$\frac{1}{\sqrt{2}}(\ket{0} - \ket{1}) = \begin{bmatrix}
		\frac{1}{\sqrt{2}} \\ 
		-\frac{1}{\sqrt{2}}\end{bmatrix}$$  
	Suppose there is $c = a + ib$ such that 
	$$c \cdot \begin{bmatrix}
			\frac{1}{\sqrt{2}} \\ 
			\frac{1}{\sqrt{2}}\end{bmatrix} 
		= \begin{bmatrix}
		\frac{1}{\sqrt{2}} \\ 
		-\frac{1}{\sqrt{2}}\end{bmatrix}$$
	Then $c \times \frac{1}{\sqrt{2}} = \frac{1}{\sqrt{2}}$
	and  $c \times \frac{1}{\sqrt{2}} = -\frac{1}{\sqrt{2}}$.
	But this is not possible, so there is a contradiction.
	Hence there is no such $c$ and these two expressions represent different 
		quantum states.
 



\item[$\textbf{e.}$] 
	$\frac{1}{\sqrt{2}}(\ket{0} - \ket{1})$ and $\frac{1}{\sqrt{2}}(\ket{1} - \ket{0})$
	\begin{align*}
		\frac{1}{\sqrt{2}}(\ket{0} - \ket{1}) &= 
		\frac{1}{\sqrt{2}}\begin{bmatrix}1 \\ -1\end{bmatrix}\\
		\frac{1}{\sqrt{2}}(\ket{1} - \ket{0}) &= 
		\frac{1}{\sqrt{2}}\begin{bmatrix}-1 \\ 1\end{bmatrix}
	\end{align*}
	Here we can take $c = -1$. 
	Then $c(\frac{1}{\sqrt{2}}\begin{bmatrix}1 \\ -1\end{bmatrix}) =  \frac{1}{\sqrt{2}}\begin{bmatrix}-1 \\ 1\end{bmatrix}$. \\
	Therefore these two expressions represent different quantum states.

\item[$\textbf{f.}$] 
	$\frac{1}{\sqrt{2}}(\ket{0} + \textbf{i}\ket{1})$ and 
	$\frac{1}{\sqrt{2}}(\textbf{i}\ket{1} - \ket{0})$
	\begin{align*}
		\frac{1}{\sqrt{2}}(\ket{0} + \textbf{i}\ket{1}) &= 
		\frac{1}{\sqrt{2}}\begin{bmatrix}1 \\ \textbf{i}\end{bmatrix} \\
		\frac{1}{\sqrt{2}}(\textbf{i}\ket{1} - \ket{0}) &= 
		\frac{1}{\sqrt{2}}\begin{bmatrix}-1 \\ \textbf{i}\end{bmatrix}
	\end{align*}
	Suppose there is $c = a + ib$ such that 
	$$c \cdot \begin{bmatrix}
			\frac{1}{\sqrt{2}} \\ 
			\frac{\textbf{i}}{\sqrt{2}}\end{bmatrix} 
		= \begin{bmatrix}
		-\frac{1}{\sqrt{2}} \\ 
		\frac{\textbf{i}}{\sqrt{2}}\end{bmatrix}$$
	Factoring out the $\frac{1}{\sqrt{2}}$, we must have $c$ such that $c \times 1 = -1$ and $c \times i = i$.\\
	So $$(a + ib)(1 + i0) = -1 \Rightarrow a + ib = -1 \Rightarrow a = -1,\ b = 0$$
	$$(a + ib)(0 + i) = i \Rightarrow -b + ia = i \Rightarrow a = 1, b = 0$$
	But this is a contradiction because $a$ cannot be both $1$ and $-1$. 
	Hence there is no such $c$ and these two expressions represent different 
		quantum states.

\newpage
\item[$\textbf{g.}$] 
	$\frac{1}{\sqrt{2}}(\ket{+} + \ket{-})$ and $\ket{0}$
	\begin{align*}
	\frac{1}{\sqrt{2}}(\ket{+} + \ket{-}) 
		&= \frac{1}{\sqrt{2}}(\frac{1}{\sqrt{2}}(\ket{0} + \ket{1}) 
		+ \frac{1}{\sqrt{2}}(\ket{0} - \ket{1})) \\
		&= \frac{1}{2}(\ket{0} + \ket{1} + \ket{0} - \ket{1}) \\
		&= \frac{1}{2}(\ket{0} + \ket{0}) \\
		&= \frac{1}{2}\begin{bmatrix}
			2\\0
			\end{bmatrix} \\
		&= \begin{bmatrix} 1 \\ 0 \end{bmatrix} \\
		&= \ket{0}
	\end{align*}
	So the two expressiones represent the same quantum states!

\item[$\textbf{h.}$] 
	$\frac{1}{\sqrt{2}}(\ket{\textbf{i}} - \ket{-\textbf{i}})$ and $\ket{1}$
	\begin{align*}
	\frac{1}{\sqrt{2}}(\ket{\bfi} - \ket{-\textbf{i}})
		&= \srtt(\srtt(\ket{0} + \bfi\ket{1}) - \srtt(\ket{0} - \bfi\ket{1})) \\
		&= \frac{1}{2}(\ket{0} + \bfi\ket{1} - \ket{0} + \bfi\ket{1}) \\
		&= \frac{1}{2}(\bfi\ket{1}  + \bfi\ket{1}) \\
		&= \bfi\ket{1} 
	\end{align*}
	So we can take $c = i$, which is modulus one, and thus the two expressions 
	represent the same quantum states. 
\newpage
\item[$\textbf{i.}$] 
	$\frac{1}{\sqrt{2}}(\ket{\textbf{i}} + \ket{-\textbf{i}})$ and 
		$\frac{1}{\sqrt{2}}(\ket{-} + \ket{+})$ \\
	Let's unpack these expressions.
	\begin{align*}
	\srtt(\ket{\bfi} + \ket{-\bfi}) 
		&= \srtt(\srtt(\ket{0} + i \ket{1}) + \srtt(\ket{0} - i \ket{1})) \\
		&= \frac{1}{2}(\ket{0} + i \ket{1} + \ket{0} - i \ket{1}) \\
		&= \frac{1}{2}(\ket{0} + \ket{0}) \\
		&= \ket{0} 
	\end{align*}
	\begin{align*}
	\srtt(\ket{-} + \ket{+})
		&= \srtt(\srtt(\ket{0} - \ket{1}) + \srtt(\ket{0} + \ket{1})) \\
		&= \frac{1}{2}(\ket{0} - \ket{1} + \ket{0} + \ket{1}) \\
		&= \frac{1}{2}(\ket{0} + \ket{0}) \\
		&= \ket{0}
	\end{align*}
	Woah! These expressions represent the same quantum states!

% \ket{+} := \srtt(\ket{0} + \ket{1})
% \ket{-} := \srtt(\ket{0} - \ket{1})
%$$\ket{\textbf{i}} := \srtt(\ket{0} + i \ket{1})$$
%$$\ket{-\textbf{i}} := \srtt(\ket{0} - i \ket{1})$$

\newpage
\item[$\textbf{j.}$] 
	$\frac{1}{\sqrt{2}}(\ket{0} + \ep\ket{1})$ and
		$\frac{1}{\sqrt{2}}(\em\ket{0} + \ket{1})$ \\
	Let's take $c = e^{\frac{-\textbf{i}\pi}{4}}$.
	It suffices to show that $c(\ket{0} + \ep\ket{1}) = (\em\ket{0} + \ket{1})$.\\
	First see that $(\ket{0} + \ep\ket{1}) = \begin{bmatrix}1 \\ \ep\end{bmatrix}$.\\
		Next, $(\em\ket{0} + \ket{1}) = \begin{bmatrix}\em \\ 1\end{bmatrix}$
	\\ Now,
	\begin{align*}
		\em \cdot \begin{bmatrix}1 \\ \ep\end{bmatrix} 
			&= \begin{bmatrix}\em \\ \em\ep  \end{bmatrix} \\
			&= \begin{bmatrix}\em \\ e^0  \end{bmatrix}\\
			&= \begin{bmatrix}\em \\ 1  \end{bmatrix}
	\end{align*}
	Therefore our choice of $c$ shows that these two expressions represent the 
	same quantum state.
\end{enumerate}

\newpage
\section{Rieffel and Polak 2.3}
Which states are superpositions with respect to the standard basis, and which
are not? For each state that is a superposition, give a basis with respect to
which it is not a superposition.

\begin{enumerate}
\item[$\textbf{a.}$] $\ket{+}$ \\
	$$\ket{+} = \srtt(\ket{0} + \ket{1}) = \srtt\ket{0} + \srtt\ket{1}$$
	This is a proper linear combination of $\ket{0}$ and $\ket{1}$, so it is a superposition wtih respect to the standard basis.
	This is not a superposition with respect to the Hadamard basis, because $\ket{+}$ is one of the basis vectors of the Hadamard basis. 

\item[$\textbf{b.}$] $\srtt(\ket{+} + \ket{-})$ \\
	We know from part $g$ in problem 2.2 that $\srtt(\ket{+} + \ket{-}) = \ket{0}$.
		And this is not a proper linear combination of $\ket{0}$ and $\ket{1}$,
		so this state is not a superposition with respect to the standard basis.

\item[$\textbf{c.}$] $\srtt(\ket{+} - \ket{-})$
	\begin{align*}
		\srtt(\ket{+} - \ket{-}) 
		&= \srtt(\srtt(\ket{0} + \ket{1}) - \srtt(\ket{0} - \ket{1})) \\
		&= \frac{1}{2}(\ket{0} + \ket{1} - \ket{0} + \ket{1}) \\
		&= \frac{1}{2}(\ket{1} + \ket{1}) \\
		&= \ket{1}
	\end{align*}
	And this is not a proper linear combination of $\ket{0}$ and $\ket{1}$,
	so this state is not a superposition with respect to the standard basis.

\newpage
\item[$\textbf{d.}$] $\frac{\sqrt{3}}{2}\ket{+} - \frac{1}{2}\ket{-}$ \\
	We have a conversion matrix: 
		$$M^{\mathcal{C}}_{\mathcal{H}} = \srtt\begin{bmatrix}1&1 \\ 1&-1\end{bmatrix}$$
	The given expression is the following vector in the Hadamard basis: 
		$$\frac{1}{2}\begin{bmatrix}\sqrt{3}\\ - 1\end{bmatrix}$$
	Now we can multiply our conversion matrix with the given vector:
	\begin{align*}
	\srtt\begin{bmatrix}1&1 \\ 1&-1\end{bmatrix}\frac{1}{2}\begin{bmatrix}\sqrt{3}\\ - 1\end{bmatrix} 
		&= \frac{1}{2\sqrt{2}} \begin{bmatrix}1&1 \\ 1&-1\end{bmatrix} \begin{bmatrix}\sqrt{3}\\ - 1\end{bmatrix} \\
			&= \frac{1}{2\sqrt{2}} \begin{bmatrix}\sqrt{3} - 1 \\ \sqrt{3} + 1\end{bmatrix}
	\end{align*}
	Given that this vector has no 0 entry, we know it is a superposition in the standard basis. \\
	Now we must find a basis in which it is not a superposition.
	Let's make a basis for which this vector is not a superposition!  \\
	Consider the following basis:
		$$\mathcal{B} = \left\{\begin{bmatrix}\frac{\sqrt{3}}{2} \\ -\frac{1}{2}\end{bmatrix}, \begin{bmatrix}\frac{1}{2} \\ \frac{\sqrt{3}}{2}\end{bmatrix} \right\}$$
	If this were a basis, then the given vector is not a superposition with respect to this basis. So it remains to verify that this is indeed a basis.
	For this, it suffices to show the two vectors are orthonormal. First, it is trivial to see that both vectors have norm 1. 
	Next we must see that the inner-product is equal to 0. 
	\begin{align*}
		\left\langle\begin{bmatrix} \frac{\sqrt{3}}{2}\\-\frac{1}{2}\end{bmatrix},\begin{bmatrix}\frac{1}{2}\\\frac{\sqrt{3}}{2}\end{bmatrix}\right\rangle 
			&= \begin{bmatrix} \frac{\sqrt{3}}{2} & -\frac{1}{2}\end{bmatrix} \begin{bmatrix}\frac{1}{2}\\\frac{\sqrt{3}}{2}\end{bmatrix} \\
				&= (\frac{\sqrt{3}}{2}\frac{1}{2}) + (-\frac{1}{2}\frac{\sqrt{3}}{2}) \\
				&= \frac{\sqrt{3}}{2} - \frac{\sqrt{3}}{2} \\
				&= 0
	\end{align*}
	Since the vectors are orthonormal, we know they are linearly independent and thus comprise a basis. 
	Therefore the original expression is not a superposition with respect to the basis $\mathcal{B}$.
	
	
	
% \ket{+} := \srtt(\ket{0} + \ket{1})
% \ket{-} := \srtt(\ket{0} - \ket{1})

\item[$\textbf{e.}$] $\srtt(\ket{\bfi} - \ket{-\bfi})$ \\
	In $2.2.h$, we learned $\srtt(\ket{\bfi} - \ket{-\bfi}) = \bfi\ket{1}$.
	And this is not a proper linear combination of $\ket{0}$ and $\ket{1}$,
		so this state is not a superposition with respect to the standard basis.

\item[$\textbf{f.}$] $\srtt(\ket{0} - \ket{1})$ \\
	This is obviously a superposition with respect to the standard basis.
	This is not a superposition with respect to the Hadamard basis, because 
		$$\srtt(\ket{0} - \ket{1}) = \ket{-}$$
\end{enumerate}

\newpage
\section{Rieffel and Polak 2.4}
Which of the states in 2.3 are superpositions with respect to the Hadamard basis,
and which are not?

\begin{enumerate}
\item[$\textbf{a.}$] $\ket{+}$ \\
	This is not a superposition with respect to the Hadamard basis. 
\item[$\textbf{b.}$] $\srtt(\ket{+} + \ket{-})$ \\
	This is a superposition with respect to the Hadamard basis.
\item[$\textbf{c.}$] $\srtt(\ket{+} - \ket{-})$ \\
	This is a superposition with respect to the Hadamard basis.
\item[$\textbf{d.}$] $\frac{\sqrt{3}}{2}\ket{+} - \frac{1}{2}\ket{-}$ \\
	This is a superposition with respect to the Hadamard basis.
% \ket{+} := \srtt(\ket{0} + \ket{1})
% \ket{-} := \srtt(\ket{0} - \ket{1})
\item[$\textbf{e.}$] $\srtt(\ket{\bfi} - \ket{-\bfi})$ \\
	In the basis $S = \{\ket{\bfi}, \ket{-\bfi}\}$, this is the vector $\srtt\begin{bmatrix}1\\-1\end{bmatrix}$. \\
	We have the following change of base matrix 
		$$M^{\mathcal{H}}_{\mathcal{S}} = \srtt\begin{bmatrix}1 + \bfi & 1 - \bfi \\ 1 - \bfi & 1 + \bfi\end{bmatrix}$$
	Now we can convert via matrix multiplication:
	\begin{align*}
	M^{\mathcal{H}}_{\mathcal{S}}\srtt\begin{bmatrix}1\\-1\end{bmatrix} 
		&= \srtt\begin{bmatrix}1 + \bfi & 1 - \bfi \\ 1 - \bfi & 1 + \bfi\end{bmatrix}\srtt\begin{bmatrix}1\\-1\end{bmatrix} \\
		&= \frac{1}{2} \begin{bmatrix}1 + \bfi & 1 - \bfi \\ 1 - \bfi & 1 + \bfi\end{bmatrix}\begin{bmatrix}1\\-1\end{bmatrix} \\
		&= \frac{1}{2}\begin{bmatrix}2\bfi \\ -2\bfi  \end{bmatrix}  \\
		&= \begin{bmatrix}\bfi \\ -\bfi  \end{bmatrix} 
	\end{align*}
	Given that the resulting vector has no 0 entry, we know this is a superposition in the Hadamard basis.

\item[$\textbf{f.}$] $\srtt(\ket{0} - \ket{1})$ \\
	We know this is equal to $\ket{-}$, so it is not a superposition with respect 
		to the Hadamard basis.
\end{enumerate}




\end{document}
