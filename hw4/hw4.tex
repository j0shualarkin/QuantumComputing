\documentclass[11pt]{article}

\usepackage{graphicx}
\usepackage{amsthm}
\usepackage{dsfont}
\usepackage{amssymb}
\usepackage{amsmath}
\usepackage{listings}
\usepackage{fancyhdr}
\pagestyle{fancy}
\usepackage{indentfirst}
\usepackage{layout}
\usepackage{hanging}
\usepackage{setspace}
\usepackage{mathtools}

\DeclarePairedDelimiter\qb{\lvert}{\rangle}

\def\tit{Quantum Programming HW4}
\def\term{February 2020}

\graphicspath{{.}}

\def\auths{Joshua Larkin}

\doublespacing

\lhead{\term}
\chead{\tit}
\rhead{\thepage}
\cfoot{}


\title{
    \vspace{2in}
    \textmd{\textbf{\tit}}\\
    \normalsize\vspace{0.1in}\small{B490 : Spring 2020 }\\
    \vspace{0.1in}\large{\textit{\auths}}
    \vspace{3in}
}

\date{}

\newcommand{\icol}[1]{
  \left(\begin{smallmatrix}#1\end{smallmatrix}\right)
}

\newcommand{\re}[1]{$\text{Re }$#1}
\newcommand{\im}[1]{$\texttt{Im }#1$}

\newcommand{\n}{\newline}
\renewcommand\headrulewidth{0.4pt}
\fancyheadoffset{0.5 cm}

\oddsidemargin 0pt
\evensidemargin 0pt
\topmargin -.3in
\headsep 20pt
%\footskip 20pt
\textheight 8.5in
\textwidth 6.25in

\setlength\topmargin{0pt}
\addtolength\topmargin{-\headheight}
\addtolength\topmargin{-\headsep}
\setlength\oddsidemargin{0pt}
\setlength\textwidth{\paperwidth}
\addtolength\textwidth{-2in}
\setlength\textheight{\paperheight}
\addtolength\textheight{-2in}

\begin{document}

%%%% TITLE PAGE
\maketitle
\pagebreak

%%%%% First content Page
% \section{Work}
% Do exercises 1.1.2 thru 1.3.12 and exercises 2.1.1 thru 2.2.7
% Programming drills are optional

%%%%%%%%%%

\section{1.1.2}
Find the value of $i^{15}$. (Hint: Calculate $i$, $i^2$, $i^3$, $i^4$, and $i^5$. Find a pattern.) 

Here we go! 
\begin{align*}
	i &= \sqrt{-1} \\
	i^2 &= -1  \\
	i^3 &= -1 \times i = -i \\
	i^4 &= (-1 \times i) \times i = -(i^2) = -(-1) = 1 \\
	i^5 &= 1 \times i = i \\
	i^{15} &= (i^5)^3 = i^3 = -1 
\end{align*}
The pattern is: $i, -1, -i,\ 1$. Then it repeats. So for $i^n$ you can take $n \mod 4$ as the index in the pattern of the value.

%%%%%%%%%%

\section{1.1.3}
Let $c_1 = -3 + i$ and $c_2 = 2 - 4i$. Calculate $c_1 + c_2$ and $c_1 \times c_2$. 
\begin{align*} 
	c_1 + c_2 &= (-3 + i) + (2 - 4i)  \\
	          &= (-3 + 2) + (1 - 4)i  \\
		  &= -1 - 3i
\end{align*}
Note: $(i \times -4i) = (i \times i) \times -4 = (-1 \times -4)$ 
\begin{align*}
	c_1 \times c_2 &= (-3 + i) \times (2 - 4i) \\
		       &= (-3 \times 2) + (-3 \times -4i)  + (i \times 2) + (i \times -4i)  \\
		       &= (-6) + 12i + 2i + 4  \\
		       &= -2 + 14i
\end{align*}

%%%%%%%%%%

\section{1.1.4}
Verify that the complex number $-1 + i$ is a solution for the polynomial equation $x^2 + 2x + 2 = 0$.
\begin{align*}
	x^2 + 2x + 2 &= (-1 + i)^2 + 2(-1 + i) + 2 \\
		     &= ((-1 + i)(-1 + i)) + (2 + 0i)(-1 + i) + (2 + 0i) \\
		     &= ((-1\times -1 + -(1 \times 1)) + (-1 \times 1 + 1 \times -1)i) + (2 + 0i)(-1 + i) + (2 + 0i) \\
		     &= ((1 + -1) + (-1 + -1)i) + (2 + 0i)(-1 + i) + (2 + 0i) \\
		     &= (0 - 2i) + (((2 \times -1) - (0 \times 1)) + ((2 \times 1) + (0 \times -1))i) + (2 + 0i) \\
		     %% (r1r2 - i1i2) + (r1i2 + i1r2)i
		     &= (0 - 2i) + (-2 + 2i) + (2 + 0i) \\
		     &= (0 - 2i) + (0 + 2i) \\
		     &= 0 + 0i \\
	             &= 0
\end{align*}

%%%%%%%%%%

\section{1.2.1} Let $c_1 = (-3,-1)$ and $c_2 = (1,-2)$. Calculate their product.
\begin{align*}
	c_1 \times c_2 &= (-3,-1)(1,-2) \\
		       &= (-3*1 - -1*-2, -3*-2 + 1*-1) \\
		       &= (-3 - 2, 6 - 1)\\
		       &= (-5, 5) \\
\end{align*}


%%%%%%%%%%
\newpage

\section{1.2.2}
Verify that multiplication of complex numbers is associative.

\begin{proof}
	Fix three complex numbers $x, y, z$. \\
	We show $x \times (y \times z) = (x \times y) \times z$, where $\times$ is complex multiplication.
\begin{align*}
	x \times (y \times z) &= (x_r, x_i) \times ((y_r, y_i) \times (z_r, z_i)) \\
			      &= (x_r, x_i) \times ((y_rz_r - y_iz_i), (y_rz_i + y_iz_r)) \\
			      &= (((x_r)(y_rz_r - y_iz_i) - (x_i)(y_rz_i + y_iz_r)) , ((x_r)(y_rz_i + y_iz_r) + (x_i)(y_rz_r - y_iz_i))) \\
			      &= (((x_r)(y_rz_r - y_iz_i) - (x_i)(y_rz_i + y_iz_r)) , ((x_r)(y_rz_i + y_iz_r) + (x_i)(y_rz_r - y_iz_i))) \\
			      &= ((x_ry_rz_r - x_ry_iz_i) - (x_iy_rz_i + x_iy_iz_r) , (x_ry_rz_i + x_ry_iz_r) + (x_iy_rz_r - x_iy_iz_i)) \\
			      &= ((x_ry_rz_r - x_ry_iz_i) - (x_iy_rz_i + x_iy_iz_r) , (x_ry_rz_i + x_ry_iz_r) + (x_iy_rz_r - x_iy_iz_i)) \\
			      &= (x_ry_rz_r + -(x_ry_iz_i) + -(x_iy_rz_i) + -(x_iy_iz_r) , x_ry_rz_i + x_ry_iz_r + x_iy_rz_r + -(x_iy_iz_i)) \\
			      &=  (x_ry_rz_r + -(x_iy_iz_r) + -(x_ry_iz_i) + -(x_iy_rz_i) , x_ry_rz_i + -(x_iy_iz_i) + x_ry_iz_r + x_iy_rz_r) \\
			      &=  ( (x_ry_rz_r - x_iy_iz_r) - (x_ry_iz_i + x_iy_rz_i) , (x_ry_rz_i - x_iy_iz_i) + (x_ry_iz_r + x_iy_rz_r)  ) \\
			      &=  ( (x_ry_r - x_iy_i)(z_r) - (x_ry_i + x_iy_r)(z_i) , (x_ry_r - x_iy_i)(z_i) + (x_ry_i + x_iy_r)(z_r)  ) \\
			      &= (x_ry_r - x_iy_i , x_ry_i + x_iy_r) \times (z_r,z_i) \\
			      &= ((x_r,x_i) \times (y_r,y_i)) \times (z_r,z_i) \\
			      &= (x \times y) \times z
\end{align*}

\end{proof}

%%%%%%%%%%
\newpage 

\section{1.2.3}
Let $c_1 = 3i$ and $c_2 = -1 - i$. Calculate $\frac{c_1}{c_2}$.
So $c_1 = (a_1, b_1) = (0, 3)$ and $c_2 = (a_2, b_2) = (-1, -1)$. \\
\begin{align*}
	a_2^2 + b_2^2 &= (-1)^2 + (-1)^2 = 2 \\
	a_1a_2 + b_1b_2 &= 0 \times -1 + 3 \times -1 = 0 + -3 = -3 \\
	a_2b_1 - a_1b_2 &= -1 \times 3 - 0 \times -1 = -3 - 0 = -3 \\
\end{align*}
Then $\frac{c_1}{c_2} = (\frac{-3}{2},\frac{-3}{2}) = (-1.5, -1.5)$

%%%%%%%%%%

\section{1.2.4}
Calculate the modulus of $c = 4 - 3i$.
$$|c| = |4 - 3i| = +\sqrt{4^2 + (-3)^2} = +\sqrt{16 + 9} = +\sqrt{25} = 5$$

%%%%%%%%%%

\newpage

\section{1.2.5}
Verify that given two arbitrary complex numbers $c_1$ and $c_2$, the following equality 
always holds: 
Prove that $|c_1||c_2| = |c_1 c_2|$.
\begin{proof}
	Fix complex numbers $c_1 = (a_1,b_1)$ and $c_2 = (a_2,b_2)$. We show $|c_1||c_2|=|c_1 c_2|$. 
	\begin{align*}
	|c_1||c_2| &= |a_1 + b_1i||a_2 + b_2i| \\
		   &= (+\sqrt{a_1^2 + b_1^2})(+\sqrt{a_2^2 + b_2^2}) \\
		   &= +\sqrt{(a_1^2 + b_1^2)(a_2^2 + b_2^2)}  \\
		   &= +\sqrt{a_1^2(a_2^2 + b_2^2) + b_1^2(b_2^2 + a_2^2)}  \\
		   &= +\sqrt{a_1^2a_2^2 + a_1^2b_2^2 + b_1^2b_2^2 + b_1^2a_2^2}  \\
		   &= +\sqrt{(a_1a_2)^2 + (a_1b_2)^2 + (b_1b_2)^2 + (b_1a_2)^2}  \\
		   &= +\sqrt{(a_1a_2 - b_1b_2)^2 + (a_1b_2 + b_1a_2)^2}  \\
		   &= |(a_1a_2 - b_1b_2) + (a_1b_2 + b_1a_2)i|  \\
		   &= |(a_1,b_1)(a_2,b_2)|  \\
		   &= |c_1c_2|
	\end{align*}
\end{proof}


%%%%%%%%%%
\newpage

\section{1.2.6}
Prove that $|c_1 + c_2| \leq |c_1| + |c_2|$. When are they equal? (Hint: square both sides.)
\begin{proof}
	Fix complex numbers $c_1 = a+bi, c_2 = c+di$. We show $|c_1 + c_2| \leq |c_1| + |c_2|$.
	Consider the square of both sides. 
	$$|c_1 + c_2|^2 = |(a+c) + (b+d)i|^2 = (\sqrt{(a+c)^2 + (b+d)^2})^2 = (a+c)^2 + (b+d)^2$$
	$$(|c_1| + |c_2|)^2 = |c_1|^2 + 2|c_1||c_2| + |c_2|^2 = (a^2 + b^2) + 2|c_1||c_2| + (c^2 + d^2)$$
	So, we must show $(a+c)^2 + (b+d)^2 \leq (a^2 + b^2) + 2|c_1||c_2| + (c^2 + d^2)$.  \\
	Recall that for any $z$, $|z| = |\overline{z}|$; and $|z_1||z_2| = |z_1z_2|$ for any $z_1, z_2$. \\
	Using these: $(a+c)^2 + (b+d)^2 \leq (a^2 + b^2) + 2|c_1 \overline{c_2}| + (c^2 + d^2)$ \\
	Then:  $a^2 + 2ac + c^2 + (b^2 + 2bd + d^2) \leq (a^2 + b^2) + 2|c_1 \overline{c_2}| + (c^2 + d^2)$. \\
	Subtracting common elements gives:  $2ac + 2bd \leq 2|c_1 \overline{c_2}|$. \\
	Dividing both sides by 2:  $ac + bd \leq |c_1||c_2|$. \\
	Square both sides (again):  $(ac)^2 + 2abcd + (bd)^2 \leq |c_1|^2|c_2|^2$. \\
	Unfolding:  $(ac)^2 + 2abcd + (bd)^2 \leq (a^2 + b^2)(c^2 + d^2)$. \\
	Algebra:  $(ac)^2 + 2abcd + (bd)^2 \leq (ac)^2 + (ad)^2 + (bc)^2 + (bd)^2$. \\
	Subtract common terms:  $2abcd \leq (ad)^2 + (bc)^2$. \\
	Move the left-hand side over:  $0 \leq (ad)^2 - 2abcd + (bc)^2$. \\
	More algebra:  $0 \leq (ad - bc)^2$ \\
	And we know this is true because $0 \leq x^2$ for any $x$.
\end{proof}


%%%%%%%%%%

\section{1.2.7}
Show that for all $c \in \mathbb{C}$ we have $c + (0,0) = (0,0) + c = c$. 
That is, $(0,0)$ is an additive identity.

\begin{proof} Fix $c \in \mathbb{C}$. We show $c + (0,0) = (0,0) + c = c$. 
\begin{align*}
	c + (0,0) &= (a,b) + (0,0) \\
	          &= (a + 0, b + 0) \\
		  &= (a,b)
\end{align*}
\begin{align*}
	(0,0) + c &= (0,0) + (a,b) \\ 
		  &= (0 + a, 0 + b) \\
		  &= (a,b)
\end{align*}

\end{proof}

%%%%%%%%%%

\section{1.2.8}
Show that for all $c \in \mathbb{C}$ we have $c \times (1,0) = (1,0) \times c = c$. 
That is, $(1,0)$ is a multiplicative identity.
\begin{proof} Fix $c \in \mathbb{C}$. We show $c \times (1,0) = (1,0) \times c = c$. 
\begin{align*}
	c \times (1,0) &= (a,b)(1,0) \\
	               &= (1\times a - b \times 0, a\times 0 + 1\times b) \\
		       &= (a,b)
\end{align*}
\begin{align*}
	(1,0) \times c &= (1,0)(a,b) \\ 
		       &= (a\times 1 - 0 \times b, 1\times b + a\times 0) \\
		       &= (a,b)
\end{align*}
\end{proof}

%%%%%%%%%%

\section{1.2.9}
Verify that multiplication by $(-1,0)$ changes the sign of the real and 
imaginary components of a complex number.

\begin{proof}
Fix a complex number $c = (a,b)$. We show $(a,b)(-1,0) = (-a,-b)$.
	\begin{align*}
		(a,b)(-1,0) &= (a \times -1 - b \times 0 , a\times 0 + b \times -1 )  \\ 
		            &= (-a - 0, 0 + -b) \\
			    &= (-a,-b) 
	\end{align*}
\end{proof}

%%%%%%%%%%

\section{1.2.10}
Show that conjugation respects addition, i.e.,
$\overline{c_1} + \overline{c_2} = \overline{c_1 + c_2}$

\begin{proof}
	Fix complex numbers $c_1, c_2$. 
	\begin{align*}
		\overline{c_1} + \overline{c_2} &= \overline{a_1 + b_1i} + \overline{a_2 + b_2i}  \\ 
						&= (a_1 - b_1i) + (a_2 - b_2i) \\
						&= (a_1 + a_2) + (-b_1 + -b_2)i \\
						&= (a_1 + a_2) - (b_1 + b_2)i \\
						&= \overline{(a_1+a_2) + (b_1+b_2)i} \\
						&= \overline{c_1 + c_2}
	\end{align*}
\end{proof}

%%%%%%%%%%

\section{1.2.11}
Show that conjugation respects multiplication, i.e.,
$\overline{c_1} \times \overline{c_2} = \overline{c_1 \times c_2}$

\begin{proof}
	Fix complex numbers $c_1, c_2$. 
	\begin{align*}
		\overline{c_1} \times \overline{c_2} 
			&= \overline{a_1 + b_1i} \times \overline{a_2 + b_2i}  \\ 
			&= (a_1 - b_1i) \times (a_2 - b_2i) \\
			&= (a_1a_2 - (-b_1)(-b_2)) + (a_1(-b_2) + (-b_1)a_2)i \\
			&= (a_1a_2 - b_1b_2) + (-a_1b_2 - b_1a_2)i \\
			&= (a_1a_2 - b_1b_2) - (a_1b_2 + b_1a_2)i \\
			&= \overline{(a_1a_2 - b1b2) + (a_1b_2 + b_1a_2)i} \\
			&= \overline{(a_1 + b_1i) \times (a_2 + b_2i)} \\
		        &= \overline{c_1 \times c_2}
	\end{align*}
\end{proof}


%%%%%%%%%%

\section{1.2.12}
Consider the operation given by flipping the sign of the real part. Is this a field isomorphism of $\mathbb{C}$? If yes, prove it. Otherwise, show where it fails.

\begin{proof}
	Define $\phi\ :\ \mathbb{C} \rightarrow \mathbb{C}$
	such that $\phi(a + bi) = -a + bi$. We show $\phi$ is not a field isomorphism. 
	The trouble arises in the case of $\phi$ respecting multiplication.  \\
	For $\phi(c_1 \times c_2)$ we have $\phi((a_1a_2 - b_1b_2) + (a_1b_2 + b_1a_2)i) = -(a_1a_2 - b_1b_2) + (a_1b_2 + b_1a_2)i$ \\
	While $\phi(c_1) \times \phi(c_2) = (-a_1 + b_1i) \times (-a_2 + b_2i) = (a_1a_2 - b_1b_2) + (a_1b_2 + b_1a_2)i$
	So the real part of $\phi(c_1 \times c_2)$ is negated, yet the real part of $\phi(c_1) \times \phi(c_2)$ is not.
	Hence they are not equal and $\phi$ is not a field isomorphism of $\mathbb{C}$.
%
%	First we show it is bijective. Then we show $\phi(c_1 + c_2) = \phi(c_1) + \phi(c_2)$ and $\phi(c_1 \times c_2) = \phi(c_1) \times \phi(c_2)$.
%	Let's fix two complex numbers $x, y$, where $x = a_1 + b_1i$ and $y = a_2 + b_2i$.
%	We begin with showing $\phi$ injective. That is, $\phi(x) = \phi(y) \Rightarrow x = y$. Assume $\phi(x) = \phi(y)$. 
%	Then $a_1 - b_1i = a_2 - b_2i$. So $a_1 = a_2$ and $- b_1$ = $- b_2$, and also $b_1 = b_2$ Therefore $a_1 + b_1i = a_2 + b_2i$.
%	For surjectivity, let's pick a complex number $z = a_z + b_zi$. We give a $x = a_x + b_xi$ such that $\phi(x) = z$.
%	That is, $\phi(x) = a_x - b_xi$. Take $a_x = a_z$ and $b_x = (-b_z)$. 
%	Then $\phi(x) = \phi(a_z - b_zi) = a_z + b_zi = z$. 
%	Now we know that our map $\phi$ is a bijection.
%
%	We show that the addition operator is respected under $\phi$. 
%	We prove $\phi(c_1 + c_2) = \phi(c_1) + \phi(c_2)$
%	Fix complex numbers $c_1, c_2$, where $c_1 = a_1 + b_1i$ and $c_2 = a_2 + b_2i$. 
%	\begin{align*}
%		\phi(c_1 + c_2) &= \phi((a_1 + b_1i) + (a_2 + b_2i)) \\
%		                &= \phi((a_1 + a_2) + (b_1 + b_2)i) \\
%		                &= -(a_1 + a_2) + (b_1 + b_2)i \\
%				&= (-a_1 + -a_2) + (b_1 + b_2)i \\
%				&= (-a_1 + b_1i) + (-a_2 + b_2i) \\
%				&= \phi(a_1 + b_1i) + \phi(a_2 + b_2i) \\
%				&= \phi(c_1) + \phi(c_2)
%	\end{align*}
%
%	Finally we show that the multiplication operator is respected under $\phi$. 
%	We prove $\phi(c_1 \times c_2) = \phi(c_1) \times \phi(c_2)$
%	Fix complex numbers $c_1, c_2$, where $c_1 = a_1 + b_1i$ and $c_2 = a_2 + b_2i$. 
%	\begin{align*}
%		\phi(c_1 \times c_2) &= \phi((a_1 + b_1i) \times (a_2 + b_2i)) \\
%				     &= \phi((a_1a_2 - b_1b_2) + (a_1b_2 + b_1a_2)i) \\
%				     &= -(a_1a_2 - b_1b_2) + (a_1b_2 + b_1a_2)i \\
%				     &= (-a_1a_2 + b_1b_2) + (a_1b_2 + b_1a_2)i \\
%			             &= ? \\
%				     &= (a_1a_2 - b_1b_2) + (-a_1b_2 + b_1-a_2)i   \\
%				     &= (-a_1-a_2 - b_1b_2) + (-a_1b_2 + b_1-a_2)i   \\
%			             &= -a_1 + b_1i \times -a_2 + b_2i \\
%			             &= \phi(a_1 + b_1i) \times \phi(a_2 + b_2i) \\
%			             &= \phi(c_1) \times \phi(c_2)
%	\end{align*}
%

\end{proof}


%%%%%%%%%%

\section{1.3.1}
Draw the complex numbers $c_1 = 2 - i$ and $c_2 = 1 + i$ in the complex plane, and add them using the parallelogram rule. 
Verify that you would get the same result as adding them algebraically (the way we learned in Section 1.2).

%%%%%%%%%%

\section{1.3.2}
Let $c_1 = 2 - i$ and $c_2 = 1 + i$. Subtract $c_2$ from $c_1$ by first drawing $-c_2$ and then adding it to $c_1$ using the parallelogram rule.


%%%%%%%%%%

\section{1.3.3}
Draw the complex number given by the polar coordinates $\rho = 3$ and $\theta = \frac{\pi}{3}$. 
Compute its Cartesian coordinates.
%% Missing drawing
\begin{align*}
	c &= a + bi  \\
	  &= (a,b)  \\
	  &= (\rho \cos \theta , \rho \sin \theta)  \\
	  &= (3\cos \frac{\pi}{3}, 3\sin \frac{pi}{3})  \\
	  &= (\frac{3}{2} ,\frac{3\sqrt{3}}{2})  \\
	  &= 1.5 + (\frac{3\sqrt{3}}{2})i
\end{align*}


%%%%%%%%%%

\section{1.3.4}
Multiply $c_1 = -2 - i$ and $c_2 = -1 - 2i$ using both the algebraic and the geometric method; verify that the results are identical.

Algebraically: 
\begin{align*}
	c_1 \times c_2 &= (-2, -1) \times (-1 -2) \\
		       &= ((-2\times -1 - -1 \times -2 , -2\times-2 + -1 \times -1) \\
		       &= (2-2,4+1)\\
		       &= (0,5) \\
\end{align*}
Now we can put $c_3 = 5i$ in polar coordinates $c_3 = (\sqrt{0^2 + (5)^2}, \tan^{-1}(\frac{0}{5})) = (5, 0)$
We hope that this result equals the geomtric multiplication of $c_1$ and $c_2$. 
First, we convert $c_1$ and $c_2$ to polar coordinates.
\begin{align*}
	c_1 &= (\sqrt{(-2)^2 + (-1)^2}, \tan^{-1}(\frac{-1}{-2}))   \\
	    &= (\sqrt{4 + 1}, \tan^{-1}(\frac{1}{2})) \\
	    &= (\sqrt{5}, \tan^{-1}(\frac{1}{2}))
\end{align*}
\begin{align*}
	c_2 &= (\sqrt{(-1)^2 + (-2)^2}, \tan^{-1}(\frac{-2}{-1}))  \\
	    &= (\sqrt{1 + 4}, \tan^{-1}(2))  \\
	    &= (\sqrt{5}, \tan^{-1}(2))
\end{align*}
Now we multiply the polar coordinate representations.
\begin{align*}
	c_1 \times c_2 &= (\sqrt{5}, \tan^{-1}(\frac{1}{2})) \times (\sqrt{5}, \tan^{-1}(2)) \\
	               &= (\sqrt{5}\sqrt{5}, \tan^{-1}(\frac{1}{2}) + \tan^{-1}(2)) \\
		       &= (5, \frac{\pi}{2})
\end{align*}
%%%%%%%%%%

\section{1.3.5}
Describe the geometric effect on the plane obtained by multiplying by a real number, i.e., the function $c \mapsto  c \times r_0$ where $r_0$ is a fixed real number.

Let $c = a + bi$. We consider $c \times r_0$ for a fixed real number $r_0$. Really, $r_0 = (r_0, 0)$ in the algebraic form of complex numbers. 
So $(a,b)\times (r_0,0) = (ar_0 - b\times0 , a\times0 + b\times r_0) = (ar_0,br_0)$.
Then $(ar_0,br_0) = (\sqrt{(ar_0)^2+(br_0)^2}, tan^{-1}(\frac{br_0}{ar_0})) = (\sqrt{(ar_0)^2+(br_0)^2}, tan^{-1}(\frac{b}{a}))$.
So the geometric effect is scaling the magnitude by $r_0$ but maintaining the same phase. 

%%%%%%%%%%

\section{1.3.6}
Describe the geometric effect on the plane obtained by multiplying by a generic complex number, i.e., the function $c \mapsto  c \times c_0$ where $c_0$ is a fixed complex number.

Fix a complex number $c_0$ in polar form $c_0 = (\rho_0,\theta_0)$. 
Now take a complex number $c$, in polar form as well, $c = (\rho,\theta)$. The result of multiplying $c\times c_0$ will be $(\rho\rho_0,\theta + \theta_0)$.
That is, the modulus of $c$ will be scaled by the modulus of $c_0$ and the angle of $c$ will be added with the angle of $c_0$. 

%%%%%%%%%%

\section{1.3.7}
Divide $2 + 2i$ by $1 - i$ using both the algebraic and the geometrical method and verify that the results are the same. 
\begin{align*}
	\frac{2+2i}{1-i} &= \frac{2+2i}{1-i} \times \frac{1+i}{1+i} \\ 
			 &= \frac{(2+2i)(1+i)}{(1-i)(1+i)}\\
			 &= \frac{(2 - 2) + (2+2)i}{(1+1) + i(1-1)}\\
			 &= \frac{(2+2)i}{2}\\
			 &= 2i
%	\frac{2+2i}{1-i} &= \frac{2\times 1 + 2\times -1}{(1)^2 + (-1)^2} + \frac{1\times 1 - 2 * -1}{(1)^2 + (-1)^2}i \\
%	                 &= \frac{2 + -2}{1 + 1} + \frac{1 - -2}{1 + 1}i \\
%	                 &= \frac{0}{2} + \frac{3}{2}i \\
	                 % &= 0 + \frac{3}{2}i
\end{align*}

Converting them to polar form: \\
$2 + 2i = (\sqrt{(2)^2 + (2)^2}, \tan^{-1}(1)) = (2\sqrt{2}, \frac{\pi}{4})$ 
and $1 - i = (\sqrt{(1)^2 + (-1)^2}, \tan^{-1}(-1)) = (\sqrt{2}, -\frac{\pi}{4})$
\begin{align*}
	\frac{(2\sqrt{2}, \frac{\pi}{4})}{(\sqrt{2}, -\frac{\pi}{4})} &= (\frac{2\sqrt{2}}{\sqrt{2}} , \frac{\pi}{4} - -\frac{\pi}{4})\\
	          &= (2, \frac{\pi}{2}) \\
\end{align*}

Now let's convert $(2,\frac{\pi}{2})$ to Cartesian form: $(2\cos{\frac{\pi}{2}},2\sin{\frac{\pi}{2}}) = (0,2) = 0 + 2i$
%%%%%%%%%%

\section{1.3.8}
Let $c = 1 - i$. Convert $c$ to polar coordinates, calculate $c^5$ and revert the answers to Cartesian coordinates. 

In polar form: $c = (\sqrt{1^2 + (-1)^2},\tan^{-1}(-1)) = (\sqrt{2},-\frac{\pi}{4})$. 
Now we calculate $c^5 = ((\sqrt{2})^5, -\frac{5\pi}{4})$.
Converting back to Cartesian form: 
\begin{align*}
  c^5 &= ((\sqrt{2})^5, -\frac{5\pi}{4})  \\ 
	&= ((\sqrt{2})^5(\cos(-\frac{5\pi}{4})), (\sqrt{2})^5(\sin (-\frac{5\pi}{4})))  \\
	&= (4\sqrt{2}(-\frac{1}{\sqrt{2}}), (4\sqrt{2}(\frac{1}{\sqrt{2}}))  \\
	&= (-4, 4)  \\
	&= -4 + 4i
\end{align*}


%%%%%%%%%%

\section{1.3.9}
Find all the cube roots of $c = 1 + i$.

First let $\rho = \sqrt{1^2 + 1^2} = \sqrt{2}$ and $\theta = \tan^{-1}(\frac{1}{1}) =\tan^{-1}(1) = \frac{\pi}{4}$

\begin{align*}
	c^{\frac{1}{3}} &= (\sqrt[3]{\rho}, \frac{1}{3}(\theta + k2\pi))  \\
			&= (\sqrt[3]{\sqrt{2}}, \frac{1}{3}(\frac{\pi}{4} + k2\pi)) \\
			&= (\sqrt[6]{2}, \frac{1}{3}(\pi(\frac{1}{4} + 2k)))
\end{align*}
Now we can take values from $k \in \{0,1,2\}$, since there are 3 cube-roots. 
$$(\sqrt[6]{2}, \frac{1}{3}(\pi(\frac{1}{4}))) = (\sqrt[6]{2}, \frac{\pi}{12})$$
$$(\sqrt[6]{2}, \frac{1}{3}(\pi(\frac{1}{4} + 2))) = (\sqrt[6]{2}, \frac{3\pi}{4})$$
$$(\sqrt[6]{2}, \frac{1}{3}(\pi(\frac{1}{4} + 4))) = (\sqrt[6]{2}, \frac{17\pi}{12})$$

%%%%%%%%%%

\section{1.3.10}
Prove $\textbf{De Moivre's formula}$: \\
$$(e^{\theta i})^n = \cos(n\theta) + i \sin(n\theta)$$

\begin{proof}
	We proceed by induction. 
	In the base case, assume $n=0$. We show $(e^{\theta i})^0 = \cos(0\times \theta) + i \sin(0\times\theta)$.
	So $(e^{\theta i})^0 = 1$ and $\cos(0) + i\sin(0) = 1 + 0 = 1$, thus the statement holds in the base case.

	In the inductive case, we first assume the statement for $n$ and we show it for $n+1$.
	This means we fix $n$ and assume $(e^{\theta i})^n = \cos(n\theta) + i \sin(n\theta)$. 
	Next we show $(e^{\theta i})^{n+1} = \cos((n+1)\theta) + i \sin((n+1)\theta)$.
	First we see that $(e^{\theta i})^{n+1} = (e^{\theta i})(e^{\theta i})^n$. 
	And we know $e^{\theta i} = \cos\theta + i\sin \theta$. Our induction hypothesis applies and we have 
	$$(e^{\theta i})^{n+1} = (\cos\theta + i\sin \theta)(\cos(n\theta) + i \sin(n\theta)) = (\cos((n+1)\theta) + i \sin((n+1)\theta))$$
\end{proof}

%%%%%%%%%%

\section{1.3.11}
Write the number $c = 3 - 4i$ in exponential form.

First we convert to polar form: 
$$\rho = \sqrt{3^2 + (-4)^2} = \sqrt{25} = 5$$  $$\theta = tan^{-1}(\frac{-4}{3}) = -tan^{-1}(\frac{4}{3})$$
And in exponential form we have $\rho e^{i\theta} = 5e^{-tan^{-1}(\frac{4}{3})i} =  5e^{-0.927i}  $

%%%%%%%%%%

\section{1.3.12}
Rewrite the law for dividing complex numbers in exponential form.

Recall the law for multiplying complex numbers in exponential form: 
$$c_1c_2 = \rho_1e^{i\theta_1}\rho_2e^{i\theta_2} = \rho_1\rho_2e^{i(\theta_1 + \theta_2)}  $$
We want to define $\frac{c_1}{c_2} = xe^{iy}$. 
So we must have $c_1 = c_2xe^{iy} = \rho_2xe^{i(\theta_2 + y)} \Rightarrow x = \frac{\rho_1}{\rho_2},\ y = \theta_1 - \theta_2$.
Thus $$\frac{\rho_1e^{i\theta_1}}{\rho_2e^{i\theta_2}} = \frac{\rho_1}{\rho_2}e^{i(\theta_1 - \theta_2)}$$
This matches our intuition, as the magnitudes are divided (inverse of multiplication) and the phases are subtrated (inverse of addition).
%%%%%%%%%%
%%%%%%% 2.1.1 - 2.2.7

\newpage

\section{2.1.1}
Add the following vectors: 
$\begin{bmatrix} 
	5 + 13i \\
	6 + 2i \\
	0.53 - 6i \\
	12
\end{bmatrix}$
+ 
$\begin{bmatrix} 
	7 - 8i \\
	4i \\
	2 \\
	9.4 + 3i	
\end{bmatrix}$
\vspace{0.2in}

$= 
\begin{bmatrix} 
	(5 + 13i) + (7 - 8i) \\
	(6 + 2i) + (4i) \\
	(0.53 - 6i) + (2) \\
	(12) + (9.4 + 3i)
\end{bmatrix}
=
\begin{bmatrix} 
	(5+7) +  (13 - 8)i \\
	6 + (2+4)i \\
	(0.53+2) - 6i \\
	(12+9.4) + 3i
\end{bmatrix}
=
\begin{bmatrix} 
	12 +  5i \\
	6 + 6i \\
	2.53 - 6i \\
	21.4 + 3i
\end{bmatrix}$


%%%%%%%%%%


\section{2.1.2}
Formally prove the associativity property of complex matrix addition.
\begin{proof}
	Fix three complex vectors $V,\ W,\ X$. We show $V + (W + X) = (V + W) + X$.
	In our equational reasoning we only need to know that addition of complex numbers is associative.
%	$(V + W)[j] = V[j] + W[j]$ where $V[j]$ is a complex number for any compelx vector $V$.
%	And also that .
\begin{align*}
	(V + (W + X))[j] &= V[j] + (W + X)[j]  \\
		        &= V[j] + (W[j] + X[j])   \\
		        &= (V[j] + W[j]) + X[j]  \\
		        &= (V + W)[j] + X[j]  \\
		        &= ((V+W)+X)[j] \\
			&= (V+W) + X
	\end{align*}
\end{proof}
%%%%%%%%%%

\section{2.1.3}
Scalar multiply $8 - 2i$ with $\begin{bmatrix}
				16 + 2.3i \\
				-7i \\
				6 \\
				5 - 4i
				\end{bmatrix} \Rightarrow  
\begin{bmatrix}
	(8-2i) \times (16 + 2.3i) \\
	(8-2i) \times (-7i) \\
	(8-2i) \times (6) \\
	(8-2i) \times (5 - 4i)
\end{bmatrix}
=
\begin{bmatrix}
	132.6 + -13.6i \\
	-14.0 - 56.0i \\
	48 - 12i \\
	32 - 42
\end{bmatrix} $

\section{2.1.4}
Formally prove that $(c_1 + c_2) \cdot V = c_1 \cdot V + c_2 \cdot V$.

\begin{proof}
	Fix complex numbers $c_1, c_2$ and a complex vector $V$. We show $(c_1 + c_2) \cdot V = c_1 \cdot V + c_2 \cdot V$
	\begin{align*}
		((c_1 + c_2) \cdot V)[j] &= (((a_1+b_1i) + (a_2+b_2i)) \cdot V)[j] \\
					 &= ((a_1+b_1i) + (a_2+b_2i)) \times V[j] \\
					 &= (a_1+b_1i) \times V[j] + (a_2+b_2i) \times V[j] \\
					 &= ((a_1 + b_1i) \cdot V)[j] + ((a_2 + b_2i) \cdot V)[j] \\
					 &= (c_1 \cdot V + c_2 \cdot V)[j]
	\end{align*}
\end{proof}
\newpage
%% BEGIN Section 2

\section{2.2.1} Let $r_1 = 2, r_2 = 3$, and $V = \begin{bmatrix}2 \\ -4 \\ 1 \end{bmatrix}$.
	Verify Property (vi), i.e., calculate $r_1 \cdot (r_2 \cdot V)$ and $(r_1 \times r_2) \cdot V$ and show that they coincide.

\begin{align*}
	r_1 \cdot (r_2 \cdot V) &= 2 \cdot (3 \cdot \begin{bmatrix}2 \\ -4 \\ 1 \end{bmatrix}) \\
		&= 2 \cdot \begin{bmatrix} 3 \times 2 \\  3 \times -4 \\  3 \times 1 \end{bmatrix} \\
		&= \begin{bmatrix} 2 \times 3 \times 2 \\  2 \times 3 \times -4 \\ 2 \times  3 \times 1 \end{bmatrix} \\
			&= (2 \times 3) \cdot  \begin{bmatrix}2 \\ -4 \\ 1 \end{bmatrix}  \\
			&= (r_1 \times r_2) \cdot V 
\end{align*}

%%%%%%%%%%%%%%%%

\section{2.2.2} Draw pictures in $\mathbb{R}^3$ that explain Properties (vi) and (viii) of the definition of a real vector space. 

Property (vi) states that scalar multiplication respects complex multiplication, i.e., $c_1 \cdot (c_2 \cdot V) = (c_1 \times c_2) \cdot V$ \\
this makes sense because you are just scaling then scaling again, which is the same as scaling the scalar then scaling the vector. 

Property (viii) states that scalar multiplication distributes over complex addition, i.e., $(c_1 + c_2) \cdot V = c_1 \cdot V + c_2 \cdot V $. 
This makes sense because you are going to stretch V by a scalar resulting from adding two scalars. 
Then you could scale the vector by one of the scalars, and separately scale it by the other. 
Then you can make a parallelgoram with these scaled values to find the addition. 

%%%%%%%%%%%%%%%%

\section{2.2.3} Let $c_1 = 2i$, $c_2 = 1 + 2i$ and 
$A = \begin{bmatrix}1 - i & 3 \\ 2+2i & 4+i  \end{bmatrix}$.  
	Verify Properties (vi) and (viii) in showing $\mathbb{C}^{2 \times 2}$ is a complex vector space.\\
	\begin{enumerate}
		\item[Prop (vi)] Scalar multiplication respects complex multiplication. \\
			We show $c_1 \cdot (c_2 \cdot A) = (c_1 \times c_2) \cdot A$ 
			\begin{align*}
				c_1 \cdot (c_2 \cdot A) &= 2i \cdot (1+2i \cdot 
				\begin{bmatrix} 1 - i & 3 \\ 2+2i & 4+i  \end{bmatrix})  \\
							&= 2i \cdot 
				\begin{bmatrix} (1+2i) \times (1 - i) & (1+2i) \times 3 \\ 
				(1+2i) \times (2+2i) & (1+2i) \times (4+i)  \end{bmatrix} \\
							&= 
							\begin{bmatrix} 2i \times (1+2i) \times (1 - i) 
								& 2i \times (1+2i) \times 3 \\ 
								2i \times (1+2i) \times (2+2i) 
							& 2i \times (1+2i) \times (4+i)  
							\end{bmatrix} \\
							&= (2i \times (1+2i)) \cdot \begin{bmatrix}1 - i & 3 \\ 2+2i & 4+i  \end{bmatrix} \\
							&= (c_1 \times c_2) \cdot A) 
			\end{align*}

		\item[Prop (viii)] Scalar multiplication distributes over complex addition. \\
			We show $(c_1 + c_2) \cdot A = c_1 \cdot A + c_2 \cdot A$
			\begin{align*}
                                (c_1 + c_2) \cdot A 
                                &= (2i + (1+2i)) \cdot 
                                \begin{bmatrix}1 - i & 3 \\ 2+2i & 4+i  \end{bmatrix} \\
                                &=   
                                \begin{bmatrix} 
                                (2i + (1+2i)) \times 1 - i & 
                                (2i + (1+2i)) \times 3 \\ 
                                (2i + (1+2i)) \times 2+2i & 
                                (2i + (1+2i)) \times 4+i  
                                \end{bmatrix} \\
                                &=   
                                \begin{bmatrix} 
                                2i \times (1-i) + (1+2i) \times (1-i) & 
                                2i \times 3 + (1+2i) \times 3 \\
                                2i \times (2+2i) + (1+2i) \times (2+2i) & 
                                2i \times (4+i) + (1+2i) \times (4+i) 
                                \end{bmatrix} \\
				&=
				\begin{bmatrix} 
                                2i \times (1-i)  & 
                                2i \times 3 \\
                                2i \times (2+2i)  & 
                                2i \times (4+i) 
                                \end{bmatrix} 
                                +
                                \begin{bmatrix}
                                	(1+2i) \times (1-i) &
                                	(1+2i) \times 3 \\
                                	(1+2i) \times (2+2i) &
                                	(1+2i) \times (4+i)
                                \end{bmatrix} \\
                                &= 2i \cdot \begin{bmatrix}1 - i & 3 \\ 2+2i & 4+i  \end{bmatrix}
                                	+ (1+2i) \cdot 
                                	\begin{bmatrix}1 - i & 3 \\ 2+2i & 4+i  \end{bmatrix} \\
                                &= c_1 \cdot A + c_2 \cdot A 
			\end{align*}
	\end{enumerate}
% \begin{bmatrix}1 - i & 3 \\ 2+2i & 4+i  \end{bmatrix}
%%%%%%%%%%%%%%%%

\section{2.2.4} Show that these operations on $\mathbb{C}^{m \times n}$ satisfy Properties (v), (vi), and (viii) of being a complex vector space.

\begin{enumerate}
\item[Property (v)] Scalar multiplication has a unit. \\ 
	Fix a complex vector $V \in \mathbb{C}^{m \times n}$.
		We give a $\mathds{1}$ so that $\mathds{1} \cdot V = V$. \\
		Let $\mathds{1}$ be $1 = (1,0)$ in Cartesian form. 
		Since this is the identity for complex multiplication, the elements of $V$ will not change. Therefore $\mathds{1}$ is a unit for scalar multiplication.
		\newpage
\item[Property (vi)] Scalar multiplication respects complex multiplication. 
	Show $c_1 \cdot (c_2 \cdot V) = (c_1 \times c_2) \cdot V$ \\
		Fix $c_1, c_2 \in \mathbb{C}$ and $V \in \mathbb{C}^{m \times n}$. 
		\begin{align*}
			c_1 \cdot (c_2 \cdot V) 
			&= (c_1 \cdot (c_2 \cdot V))[j,k] \\
			&= c_1 \times (c_2 \cdot V)[j,k] \\
			&= c_1 \times c_2 \times V[j,k] \\
			&= (c_1 \times c_2) \times V[j,k] \\
			&= ((c_1 \times c_2) \cdot V)[j,k] \\
			&= (c_1 \times c_2) \cdot V
		\end{align*}

\item[Property (viii)] Scalar multiplication distributes over complex addition. 
	Show $(c_1 + c_2) \cdot V = c_1 \cdot V + c_2 \cdot V$
		Fix $c_1, c_2 \in \mathbb{C}$ and $V \in \mathbb{C}^{m \times n}$. 
		\begin{align*}
			(c_1 + c_2) \cdot V 
			&= ((c_1 + c_2) \cdot V)[j,k] \\
			&= (c_1 + c_2) \times V[j,k] \\
			&= c_1 \times V[j,k] + c_2 \times V[j,k]\\
			&= (c_1 \cdot V)[j,k] + (c_2 \cdot V)[j,k]\\
			&= ((c_1 \cdot V) + (c_2 \cdot V))[j,k] \\ 
			&= (c_1 \cdot V) + (c_2 \cdot V)
		\end{align*}



\end{enumerate}

%%%%%%%%%%%%%%%%
\newpage

\section{2.2.5} Find the transpose, conjugate, and adjoint of 
$\begin{bmatrix}
	6 - 3i & 2 + 12i & -19i \\
	0 & 5 + 2.1i & 17 \\
	1 & 2 + 5i & 3 - 4.5i 
\end{bmatrix}$

\begin{enumerate}
\item[Transpose]
$\begin{bmatrix}
	6 - 3i   & 0        & 1 \\
	2 + 12i  & 5 + 2.1i & 2 + 5i  \\
	-19i     & 17  & 3 - 4.5i 
\end{bmatrix}$
		\vspace{0.3in}
\item[Conjugate]
$\begin{bmatrix}
	6 + 3i & 2 - 12i & 19i \\
	0 & 5 - 2.1i & 17 \\
	1 & 2 - 5i & 3 + 4.5i 
\end{bmatrix}$

		\vspace{0.3in}
\item[Adjoint]
$\begin{bmatrix}
	6 + 3i   & 0        & 1 \\
	2 - 12i  & 5 - 2.1i & 2 - 5i  \\
	19i     & 17  & 3 + 4.5i 
\end{bmatrix}$
\end{enumerate}

%%%%%%%%%%%%%%%%

\section{2.2.6} Prove that conjugation respects scalar multiplication; i.e., $\overline{c \cdot A} = \overline{c} \cdot \overline{A}$

\begin{proof}
	Fix a complex number $c = a+bi$ and a vector $A$ over $\mathbb{C}$. 
	Let $c_A = c+di$ be an arbitrary element of $A$. It suffices to show $\overline{c\times c_A} = \overline{c} \times \overline{c_A}$
	\begin{align*}
		\overline{c \times c_A} 
		&= \overline{a+bi \times c+di} \\
		&= \overline{(ac-bd) + (ad + bc)i} \\
		&= (ac-bd) - (ad + bc)i \\
		&= (a-bi)(c-di) \\
		&= \overline{a+bi} \times \overline{c+di} \\
		&= \overline{c} \times \overline{c_A}
	\end{align*}
\end{proof}

%%%%%%%%%%%%%%%%

\section{2.2.7} Prove Properties (vii), (viii), and (ix) using Properties (i)-(vi).

Recall that the adjoint of $A$, denoted as $A^{\dagger}$ is defined as either
$$A^{\dagger} = (\overline{A})^T = \overline{(A^T)}$$

\begin{enumerate}
	\item[Property (i)] Transpose is idempotent: 
		$(A^T)^T = A$
	\item[Property (ii)] Transpose respects addition: 
		$(A+B)^T = A^T + A^T$
	\item[Property (iii)] Transpose respects scalar multiplication: 
		$(c \cdot A)^T = c \cdot A^T$
	\item[Property (iv)] Conjugate is idempotent: 
		$\overline{\overline{A}} = A$
	\item[Property (v)] Conjugate respects addition:
		$\overline{(A+B)} = \overline{A} + \overline{B}$
	\item[Property (vi)] Conjugate respects scalar multiplication: 
		$\overline{c \cdot A} = \overline{c} \cdot \overline{A}$
	\item[Property (vii)] Adjoint is idempotent: 
		$(A^{\dagger})^{\dagger} = A$ 
\begin{proof}
	Fix a vector $A \in \mathbb{C}^{n \times n}$. We will use both definitions of 
	adjoint to show $(A^{\dagger})^{\dagger} = A$. 
	\begin{align*}
		(A^{\dagger})^{\dagger} &= ((\overline{A})^T)^{\dagger}  \\
		                        &= \overline{(((\overline{A})^T)^T)} \\
					&= \overline{(\overline{A})} & (\texttt{Property i}) \\
					&= A  & (\texttt{Property iv})
	\end{align*}
	\end{proof}
		
	\newpage

	\item[Property (viii)] Adjoint respects addition: 
		$(A+B)^{\dagger} = A^{\dagger} + B^{\dagger}$
	\begin{proof}
		Fix vectors $A,B \in \mathbb{C}^{n \times n}$. We show  
		$(A+B)^{\dagger} = A^{\dagger} + B^{\dagger}$
		\begin{align*}
			(A+B)^{\dagger} &= \overline{(A+B)^T} \\
					&= \overline{(A^T + B^T)} & (\texttt{Property ii}) \\
					&= \overline{(A^T)} + \overline{(B^T)}  & (\texttt{Property v}) \\ 
					&= A^{\dagger} + B^{\dagger}
		\end{align*}
	\end{proof}
	\item[Property (ix)] Adjoint relates to scalar multiplication: 
		$(c \cdot A)^{\dagger} = \overline{c} \cdot A^{\dagger}$
	\begin{proof}
		Fix a complex number $c$ and a vector $A \in \mathbb{C}^{n \times n}$. 
		\begin{align*}
			(c \cdot A)^{\dagger} &= (\overline{c \cdot A})^T \\
			&= (\overline{c} \cdot \overline{A})^T & (\texttt{Property vi}) \\
			&= \overline{c} \cdot \overline{A}^T & (\texttt{Property iii})  \\
			&= \overline{c} \cdot A^{\dagger}
		\end{align*}
	\end{proof}
\end{enumerate}

%%%%%%%%%%%%%%%%

\end{document}
