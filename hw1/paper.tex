\documentclass[11pt]{article}

\usepackage{listings}
\usepackage{fancyhdr}
\pagestyle{fancy}
\usepackage{indentfirst}
\usepackage{layout}
\usepackage{hanging}
\usepackage{setspace}
\usepackage{mathtools}

\DeclarePairedDelimiter\qb{\lvert}{\rangle}

\def\tit{Quantum Programming HW1}
\def\term{January 2020}
\def\auths{Joshua Larkin}


\doublespacing

\lhead{\term}
\chead{\tit}
\rhead{\thepage}
\cfoot{}


\title{
    \vspace{2in}
    \textmd{\textbf{\tit}}\\
    \normalsize\vspace{0.1in}\small{B490 : Spring 2020 }\\
    \vspace{0.1in}\large{\textit{\auths}}
    \vspace{3in}
}

\date{}


\newcommand{\icol}[1]{
  \left(\begin{smallmatrix}#1\end{smallmatrix}\right)
}


\newcommand{\n}{\newline}
\renewcommand\headrulewidth{0.4pt}
\fancyheadoffset{0.5 cm}

\oddsidemargin 0pt
\evensidemargin 0pt
\topmargin -.3in
\headsep 20pt
%\footskip 20pt
\textheight 8.5in
\textwidth 6.25in

\setlength\topmargin{0pt}
\addtolength\topmargin{-\headheight}
\addtolength\topmargin{-\headsep}
\setlength\oddsidemargin{0pt}
\setlength\textwidth{\paperwidth}
\addtolength\textwidth{-2in}
\setlength\textheight{\paperheight}
\addtolength\textheight{-2in}

\begin{document}

%%%% TITLE PAGE
\maketitle
\pagebreak

%%%%% First content Page
\section{Response}


Mermin is interested in pushing readers to reconsider what it means for them to believe something.
Writing for a philosophy journal, Mermin writes with anticipation for people to wonder
and challenge the results of the experiment. Mermin proceeds empirically, following a
simple process of trying to explain how the machine might be working. Only, the more
we consider the configurations of the parts and the particles themselves, the farther
we move away from a solid understanding.
With just one assumption, that the particles have
one of eight possible configurations, we learn that the results contradict what we can
prove about the same color flashing and settings of the boxes. 
This is a bit bewildering because a proof leaves us without
a particular reason to believe in the results of this particular experiment.


The first result is that the two indicators always flash the same color when their switches have the same setting. 
The second, and arguably more exciting, result is when the switches are in different settings, they flash the same 
color in $\frac{1}{4}$ of the runs, and flash different colors in the remaining $\frac{3}{4}$ of the runs.
Considering an explanation, it seems some finite explanation must make sense of the results. Yet, when one actually
formalizes the problem, they find that the split should be $\frac{1}{3}$ and $\frac{2}{3}$, respectively. 
And these probabilities are figured after running the experiment a large amount of the time. These latter figures are 
computed by considering that the particles in the experiment are of a certain amount of possible states.
However, this very asssumption is untenable because of the difference in results and calculated expectations. 
It seems the more one tries to understand the problem, the less they do. A different approach is to try and 
understand how one case could happen, but it seems when one finds such explanation,
explaining the other case under the same assumptions is still difficult.
Hence, the results of the machine are completely abnormal 
in some sense, and one is left to wonder only about what they do not know. 

Mermin discusses possible connections within the device such as hidden wires, but there is a more abstract notion of connectedness. 
"Physicists live with the existence of the device by implicitly denying the absence of connections between its pieces" (Mermin, p.407). 
Just as a standard mathematical explanation does not suffice, the standard physicists would also presume some sort of 
completeness regarding the pieces of the machine and its results. Mermin suggets that the absence of connections haunts people
so much so that they quit trying to understand the machine without connections, and instead hold on to a more abstract principle about 
there being undiscernable connections. This concept shows the magic of the experiment, because certain observers will think 
that there is something mysterious at play that they do not understand. Perhaps they belive they could understand it if only
given more information. Or Mermin's machine is in a rare category of spectacles that actually seems to reach into an 
unknown world and provide results that are not to come with understanding.
The result is an instance where nothing 
about it suggests there is a connectedness between parts in configuration or theory, yet on observation we are left thinking
there must be. 

I wonder about the impact of perception with regard to the implications of the experiment. How observation and mechanics that produce results
are not exactly linked in the same way most people would believe. This is an attempt at saying there are more levels to why something happened 
than just the immediately perceivable ones. Yet it seems to be a quality of humans, that we gather as much information as we feel necessary
before we are prepared to make a stance or form a belief about why the thing happened. I think this is unavoidable in a certain sense, and with 
practice, people are able to recognize one thought process and leave it for a certain in-the-moment presence. The experiment suggets to me 
that this ability, the in-the-moment presence, has more to it than we appreciate or understand. Partly because it "swims up stream" compared to
our usual analytic and judgmental mindsets. But also because we spend less time doing this. These ideas usually float around like daydreams or 
assertions in contexts other than physics/computer science/philosophy, but it would seem quantum mechanics would direct us to reconsider
what we are looking for when we observe something. 

%%%%% Bibliography
\newpage
\section{Bibliography}  % \cite{name of entry}
\begin{hangparas}{.25in}{1}
N. David Mermin. Quantum mysteries for anyone. \textit{Journal of Philosophy} 78, 7 (Jul. 1981), 397-408

\end{hangparas}

\end{document}
