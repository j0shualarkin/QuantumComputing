\documentclass[11pt]{article}

\usepackage{graphicx}
\usepackage{amsthm}
\usepackage{dsfont}
\usepackage{amssymb}
\usepackage{amsmath}
\usepackage{listings}
\usepackage{fancyhdr}
\pagestyle{fancy}
\usepackage{indentfirst}
\usepackage{layout}
\usepackage{hanging}
\usepackage{setspace}
\usepackage{mathtools}
\usepackage{physics}

\DeclarePairedDelimiter\qb{\lvert}{\rangle}

\def\tit{Quantum Programming HW7}
\def\term{March 2020}

\graphicspath{{.}}

\def\auths{Joshua Larkin}

\doublespacing

\lhead{\term}
\chead{\tit}
\rhead{\thepage}
\cfoot{}

\title{
    \vspace{2in}
    \textmd{\textbf{\tit}}\\
    \normalsize\vspace{0.1in}\small{B490 : Spring 2020 }\\
    \vspace{0.1in}\large{\textit{\auths}}
    \vspace{3in}
}

\date{}

\newcommand{\icol}[1]{
  \left(\begin{smallmatrix}#1\end{smallmatrix}\right)
}

\def\aket{\ket{\alpha}}
\def\bket{\ket{\beta}}
\def\haf{\frac{1}{2}}
\def\bfi{\ensuremath\textbf{i}}
\def\srtt{\ensuremath\frac{1}{\sqrt{2}}}
\def\em{\ensuremath e^{\frac{-\textbf{i}\pi}{4}}}
\def\ep{\ensuremath e^{\frac{\textbf{i}\pi}{4}}}

\def\bpp{\ket{\Phi^{+}}}
\def\bpm{\ket{\Phi^{-}}}
\def\bsp{\ket{\Psi^{+}}}
\def\bsm{\ket{\Psi^{-}}}

\renewcommand\headrulewidth{0.4pt}
\fancyheadoffset{0.5 cm}

\oddsidemargin 0pt
\evensidemargin 0pt
\topmargin -.3in
\headsep 20pt
%\footskip 20pt
\textheight 8.5in
\textwidth 6.25in

\setlength\topmargin{0pt}
\addtolength\topmargin{-\headheight}
\addtolength\topmargin{-\headsep}
\setlength\oddsidemargin{0pt}
\setlength\textwidth{\paperwidth}
\addtolength\textwidth{-2in}
\setlength\textheight{\paperheight}
\addtolength\textheight{-2in}

\begin{document}

%%%% TITLE PAGE
\maketitle
\pagebreak

%%%%% First content Page

%%%%%%%%%%%%%%%%%%%%%%%%%%%%%%%%%%%%%%%%%%%%%%%%%%%%%%%%%%%%
\section*{Exercise 3.2}

Show by example that a linear combination of entangled states is 
not necessarily entangled.

\begin{proof}
    Let $V = \{\ket{00},\ket{01},\ket{10},\ket{11}\}$ be the underlying vector space of a quantum state space. 
    Consider the following tensor decomposition of $V$: $\ket{00}  \otimes \ket{01} \otimes \ket{10} \otimes \ket{11}$.\\
    Let $\ket{\alpha} \in V$ such that $\ket{\alpha} = \srtt(\ket{00} + \ket{11})$
    and $\ket{\beta} \in V$ such that $\ket{\beta} = \srtt(\ket{01} + \ket{10})$.  \\
    We show that both $\ket{\alpha}$ and $\ket{\beta}$ are entangled with respect to the given tensor decomposition,
    but a linear combination of them is not entangled.
    \begin{enumerate}
        \item[$\textbf{a.}$] $\ket{\alpha}$ entangled \\
            Suppose towards a contradiction that $\aket$ is unentangled with respect to the given decomposition. \\
            That is, 
            \begin{align*}
                \srtt(\ket{00} + \ket{11}) 
                &= (c_1\ket{0} + d_1\ket{1}) \otimes (c_2\ket{0} + d_2\ket{1}) \\
                &= c_1c_2\ket{00} + c_1d_2\ket{01} + d_1c_2\ket{10} + d_1d_2\ket{11} 
            \end{align*} 
        From this, we have the following equations: 
            \begin{align*}
                c_1c_2 &= \srtt & c_1d_2 = 0 \\
                d_1c_2 &= 0  & d_1d_2 = \srtt 
            \end{align*} 
            Because we are in an integral domain, $c_1d_2 = 0$ implies that either $c_1 = 0$ or $d_2 = 0$.
            But this is a contradiction since $c_1c_2 = \srtt$ and $d_1d_2 = \srtt$.  \\
        Therefore $\ket{\alpha}$ is entangled with respect to the given decomposition.
\newpage
        \item[$\textbf{b.}$] $\bket$ entangled \\
        Suppose towards a contradiction that $\bket$ is unentangled with respect to the given decomposition. \\
            That is, 
            \begin{align*}
                \srtt(\ket{01} + \ket{10}) 
                &= (c_1\ket{0} + d_1\ket{1}) \otimes (c_2\ket{0} + d_2\ket{1}) \\
                &= c_1c_2\ket{00} + c_1d_2\ket{01} + d_1c_2\ket{10} + d_1d_2\ket{11} 
            \end{align*} 
        From this, we have the following equations: 
            \begin{align*}
                c_1c_2 &= 0 & c_1d_2 = \srtt \\
                d_1c_2 &= \srtt  & d_1d_2 = 0
            \end{align*} 
            Because we are in an integral domain, $c_1c_2 = 0$ implies that either $c_1 = 0$ or $c_2 = 0$.
            But this is a contradiction since $d_1c_2 = \srtt$ and $c_1d_2 = \srtt$.  \\
        Therefore $\bket$ is entangled with respect to the given decomposition.



        \item[$\textbf{c.}$] $\aket + \bket$ not entangled \\
            We can take scalars of 1 for our linear combination and consider $\aket + \bket$.
            \begin{align*}
                \aket + \bket &= \srtt(\ket{00} + \ket{11}) + \srtt(\ket{01} + \ket{10}) \\
                &= \srtt(\ket{00} + \ket{11} + \ket{01} + \ket{10}) \\
                &= \srtt\ket{00} + \srtt\ket{11} + \srtt\ket{01} + \srtt\ket{10}
            \end{align*}
        Which is obviously not an entangled state with respect to the decomposition we gave.
    \end{enumerate}
\end{proof}
\newpage

%%%%%%%%%%%%%%%%%%%%%%%%%%%%%%%%%%%%%%%%%%%%%%%%%%%%%%%%%%%%
\section*{Exercise 3.3}

Show that the state 

$$\ket{W_n} = \frac{1}{\sqrt{n}}(\ket{0\dots001} + \ket{0\dots010} + \ket{0\dots100} 
+ \cdots + \ket{1\dots000})$$

is entangled, with respect to the decomposition into the $n$ qubits, for every $n > 1$.

\begin{proof}
    Note: when we say a state is entangled in this proof, we mean entangled with respect to the decomposition into the $n$ qubits. 
    Fix $n$ such that $n>1$. \\ Suppose, towards a contradiction, that $\ket{W_n}$ is unentangled. \\
    Then we can write $\ket{W_n} = (c_1\ket{0} + d_1\ket{1}) \otimes (c_2\ket{0} + d_2\ket{1}) \otimes \cdots \otimes (c_n\ket{0} + d_n\ket{1})$
    The resulting of the tensor products is very large and is of the form: 
    $$c_1c_2\dots c_n\ket{0\dots000} + c_1d_2\dots c_n\ket{010\dots0} + \cdots + d_1\dots d_n\ket{1\dots111}$$
    Examining the first two terms tells us that $c_1c_2\dots c_n = 0$ and $c_1d_2\dots c_n = \frac{1}{\sqrt{n}}$. \\
    Since we are in an integral domain, we must have that each $c_i = 0$. \\
    Yet, $c_1d_2\dots c_n = \frac{1}{\sqrt{n}}$, so we must have a contradiction.  \\
    Therefore $\ket{W_n}$ is entangled with respect to the decomposition into $n$ qubits, for every $n > 1$.
\end{proof}
\newpage
%%%%%%%%%%%%%%%%%%%%%%%%%%%%%%%%%%%%%%%%%%%%%%%%%%%%%%%%%%%%
\section*{Exercise 3.4}

Show that the state 

$$\ket{GHZ_n} = \srtt(\ket{00\dots0} + \ket{11\dots1})$$

is entangled, with respect to the decomposition into the $n$ qubits, for every $n > 1$.


\begin{proof}
    Note: when we say a state is entangled in this proof, we mean entangled with respect to the decomposition into the $n$ qubits. 
    Fix $n$ such that $n>1$. \\ Suppose, towards a contradiction, that $\ket{GHZ_n}$ is unentangled. \\
    Then we can write $\ket{GHZ_n} = (c_1\ket{0} + d_1\ket{1}) \otimes (c_2\ket{0} + d_2\ket{1}) \otimes \cdots \otimes (c_n\ket{0} + d_n\ket{1})$
    The resulting of the tensor products is very large and is of the form: 
    $$c_1c_2\dots c_n\ket{0\dots000} + c_1d_2\dots c_n\ket{010\dots0} + \cdots + d_1\dots d_n\ket{1\dots111}$$
    Examining the first two terms tells us that $c_1c_2\dots c_n = \srtt$ and $c_1d_2\dots c_n = 0$. \\
    Since we are in an integral domain, we must have that $d_2 = 0$. \\
    Yet, $d_1d_2\dots d_n = \srtt$, so we must have a contradiction.  \\
    Therefore $\ket{GHZ_n}$ is entangled with respect to the decomposition into $n$ qubits, for every $n > 1$.
\end{proof}

%%%%%%%%%%%%%%%%%%%%%%%%%%%%%%%%%%%%%%%%%%%%%%%%%%%%%%%%%%%%
\section*{Exercise 3.5}

Is the state $\srtt(\ket{0}\ket{+} + \ket{1}\ket{-})$ entangled?

This question does not make sense because it does not give a decomposition relative to the state.

%%%%%%%%%%%%%%%%%%%%%%%%%%%%%%%%%%%%%%%%%%%%%%%%%%%%%%%%%%%%
\section*{Exercise 3.6}

If someone asks you whether the state $\ket{+}$ is entangled, what will you say?
\\
That doesn't make sense, you must ask if the state is entangled relative to a decomposition of a tensor product space.

%%%%%%%%%%%%%%%%%%%%%%%%%%%%%%%%%%%%%%%%%%%%%%%%%%%%%%%%%%%%
\section*{Exercise 3.7}

Write the following states in terms of the Bell basis.

\begin{itemize}
    \item[$\textbf{a.}$] $\ket{00}$  \\
        We can add the Bell states $\ket{\Phi^+}$ and $\ket{\Phi^-}$, scaling both by $\srtt$.
        \begin{align*}
         \srtt\ket{\Phi^+} + \srtt\ket{\Phi^-} 
            &= \srtt(\srtt(\ket{00} + \ket{11})) + \srtt(\srtt(\ket{00} - \ket{11})) \\
            &= \haf(\ket{00} + \ket{11} + \ket{00} - \ket{11}) \\
            &= \haf(2\ket{00}) \\
            &= \ket{00}
        \end{align*} 

    \item[$\textbf{b.}$] $\ket{+}\ket{-}$
        \begin{align*}
            \ket{+}\ket{-} 
            &= \srtt(\ket{0} + \ket{1}) \otimes \srtt(\ket{0} - \ket{1}) \\
            &= \srtt\srtt[(\ket{0} + \ket{1}) \otimes (\ket{0} - \ket{1})] \\
            &= \frac{1}{2}(\ket{00} + \ket{10} - \ket{01} - \ket{11}) \\
            &= \haf(\ket{00} - \ket{11}) + \haf(\ket{10} - \ket{01}) \\
            &= \srtt\srtt(\ket{00} - \ket{11}) + \srtt\srtt(\ket{10} - \ket{01}) \\
            &= \srtt\ket{\Phi^{-}} - \srtt\ket{\Psi^{-}}
        \end{align*}
        The last step folows from the fact that $-\ket{\Psi^{-}} = -\srtt(\ket{01} - \ket{10}) = \srtt(\ket{10} - \ket{01})$. \\
        Therefore $\ket{+}\ket{-} = \srtt\ket{\Phi^{-}} - \srtt\ket{\Psi^{-}}$

\newpage

    \item[$\textbf{c.}$] $\frac{1}{\sqrt{3}}(\ket{00} + \ket{01} + \ket{10})$  \\
        From the first part of this exercise we know $\ket{00} = \srtt(\bpp + \bpm)$.  \\
        We claim $\ket{01} = \srtt(\bsp + \bsm)$ and $\ket{10} = \srtt(\bsp - \bsm)$. 
        \begin{align*}
            \srtt(\bsp + \bsm) &= \srtt(\srtt(\ket{01} + \ket{10}) + \srtt(\ket{01} - \ket{10})) \\
            &= \haf((\ket{01} - \ket{10}) + (\ket{01} - \ket{10})) \\
            &= \haf(2\ket{01}) \\
            &= \ket{01} \\
            \srtt(\bsp - \bsm) &= \srtt(\srtt(\ket{01} + \ket{10}) - \srtt(\ket{01} - \ket{10})) \\
            &= \haf((\ket{01} + \ket{10}) - (\ket{01} - \ket{10})) \\
            &= \haf(\ket{01} + \ket{10} - \ket{01} + \ket{10}) \\
            &= \haf(2\ket{10}) \\
            &= \ket{10} 
        \end{align*}
        Therefore the given state can be written as $$\frac{1}{\sqrt{3}}(\srtt(\bpp + \bpm) + \srtt(\bsp + \bsm) + \srtt(\bsp - \bsm))$$
        We can simplify this to  
        $$\frac{1}{\sqrt{3}}\srtt(\bpp + \bpm + 2\bsp)$$
        Now we will verify this result:
        \begin{align*}
            (\bpp + \bpm + 2\bsp)  % dont forget: \frac{1}{\sqrt{3}}\srtt on the outside
            &= (\srtt(\ket{00} + \ket{11}) + \srtt(\ket{00} - \ket{11}) + 2(\srtt(\ket{01} + \ket{10}))) \\
            &= \srtt(\ket{00} + \ket{11} + \ket{00} - \ket{11} + 2\ket{01} + 2\ket{10}) \\
            &= \srtt(2\ket{00} + 2\ket{01} + 2\ket{10}) & (A) \\
            \frac{1}{\sqrt{3}}\srtt(A) &= \frac{1}{\sqrt{3}}\srtt(\srtt(2\ket{00} + 2\ket{01} + 2\ket{10})) \\
            &= \frac{1}{\sqrt{3}}\haf((2\ket{00} + 2\ket{01} + 2\ket{10})) \\
            &= \frac{1}{\sqrt{3}}(\ket{00} + \ket{01} + \ket{10}) 
        \end{align*}
\end{itemize}

%%%%%%%%%%%%%%%%%%%%%%%%%%%%%%%%%%%%%%%%%%%%%%%%%%%%%%%%%%%%
\section*{Exercise 3.8}

\begin{itemize}
    \item[$\textbf{a.}$] Show that $\srtt(\ket{0}\ket{0} + \ket{1}\ket{1})$ 
        and $\srtt(\ket{+}\ket{+} + \ket{-}\ket{-})$ refer to the same 
        quantum state. 
    \begin{proof}
        We know $\srtt(\ket{0}\ket{0} + \ket{1}\ket{1}) = \srtt(\ket{00} + \ket{11})$.
        Let's see about the other:
        \begin{align*}
            \ket{+}\ket{+} &= (\srtt(\ket{0} + \ket{1})) \otimes
            (\srtt(\ket{0} + \ket{1}))
            \\
            &= \haf((\ket{0} + \ket{1}) \otimes (\ket{0} + \ket{1})) \\
            &= \haf(\ket{00} + \ket{10} + \ket{01} + \ket{11})\\ %% end ++
            \ket{-}\ket{-} &= (\srtt(\ket{0} - \ket{1})) \otimes (\srtt(\ket{0} - \ket{1})) \\
            &= \haf((\ket{0} - \ket{1})\otimes (\ket{0} - \ket{1})) \\
            &= \haf(\ket{00} - \ket{10} - \ket{01} + \ket{11}) \\ %% end --
            \ket{+}\ket{+} + \ket{-}\ket{-}  &= \haf((\ket{00} + \ket{10} + \ket{01} + \ket{11}) + (\ket{00} - \ket{10} - \ket{01} + \ket{11}))  \\
            &= \haf(2\ket{00} + 2\ket{11}) \\
            &= \ket{00} + \ket{11} \\
            \srtt(\ket{+}\ket{+} + \ket{-}\ket{-}) &= \srtt(\ket{00} + \ket{11})
        \end{align*}
        There we have it! So, $\srtt(\ket{0}\ket{0} + \ket{1}\ket{1}) = \srtt(\ket{+}\ket{+} + \ket{-}\ket{-})$ and are equivalent quantum states.
    \end{proof}
    
\item[$\textbf{b.}$] Show that $\srtt(\ket{0}\ket{0} - \ket{1}\ket{1})$
        refers to the same state as
        $\srtt(\ket{\bfi}\ket{\bfi} + \ket{-\bfi}\ket{-\bfi})$
    \begin{proof}
        \begin{align*}
        \ket{\bfi}\ket{\bfi} 
            &= \srtt(\ket{0} + i\ket{1}) \otimes \srtt(\ket{0} + i\ket{1}) \\
            &= \haf((\ket{0} + i\ket{1}) \otimes (\ket{0} + i\ket{1})) \\
            &= \haf(\ket{00} + i\ket{10} + i\ket{01} - \ket{11})  %% end pos
        \end{align*}
        \begin{align*}
        \ket{-\bfi}\ket{-\bfi} 
            &= \srtt(\ket{0} - i\ket{1}) \otimes \srtt(\ket{0} - i\ket{1}) \\ 
            &= \haf((\ket{0} - i\ket{1}) \otimes (\ket{0} - i\ket{1})) \\ 
            &= \haf(\ket{00} - i\ket{10} - i\ket{01} - \ket{11}) %% end neg
        \end{align*}
        \begin{align*}
        \ket{\bfi}\ket{\bfi} + \ket{-\bfi}\ket{-\bfi} 
            &= \haf(\ket{00} + i\ket{10} + i\ket{01} - \ket{11}) + \haf(\ket{00} - i\ket{10} - i\ket{01} + \ket{11}) \\
            &= \haf(2\ket{00} - 2\ket{11})   \\
            &= \ket{00} - \ket{11}
        \end{align*}
        So $\srtt(\ket{\bfi}\ket{\bfi} + \ket{-\bfi}\ket{-\bfi}) = \srtt(\ket{00} - \ket{11}) = \srtt(\ket{0}\ket{0} - \ket{1}\ket{1})$. \\
        Thus the two states are equivalent.
    \end{proof}
\end{itemize}

%%%%%%%%%%%%%%%%%%%%%%%%%%%%%%%%%%%%%%%%%%%%%%%%%%%%%%%%%%%%
\newpage
\section*{Exercise 3.9b}

Show that this representation is unique in the sense that any two different vectors
of this form represent different quantum states. 
\begin{proof}
    Fix a number $n$ of qubits. Let's take two different vectors of the given form:
$$s = b_{0}\ket{0\dots00} + b_{1}\ket{0\dots01} + \cdots + b_{2^{n}-1}\ket{1\dots11}$$
$$t = c_{0}\ket{0\dots00} + c_{1}\ket{0\dots01} + \cdots + c_{2^{n}-1}\ket{1\dots11}$$
where the first non-zero $b_i$ and $c_i$ are real and non-negative.
We proceed by the contrapositive. That is, we assume the two states are equivlanet, and we show that they are equal as vectors.
So we assume there is a modulus 1 complex number $z$ such that 
    $$z(b_{0}\ket{0\dots00} + b_{1}\ket{0\dots01} + \cdots + b_{2^{n}-1}\ket{1\dots11})
    = c_{0}\ket{0\dots00} + c_{1}\ket{0\dots01} + \cdots + c_{2^{n}-1}\ket{1\dots11} $$
    Since $z$ is modulus 1, we have $z = e^{i\theta}$, in exponential form. 
    We may assume without loss of generality that the first coefficient $b_0$ is real and non-negative. 
    So when we multiply by z, we have $zb_0 = c_0$.
    \begin{align*}
        zb_0 &= e^{i\theta}b_0 \\
        &= b_0e^{i\theta} \\
        &= c_0 \\
        &\Rightarrow b_0 = c_0
    \end{align*}
    In this process we assumed that $\theta = 0$, but technically it could be $-1$. However, our assumption 
    that the first non-zero coefficient is real and positive, so $-1$ is not a possible value for theta.
    Therefore, the two vectors are equal. 
\end{proof}

%%%%%%%%%%%%%%%%%%%%%%%%%%%%%%%%%%%%%%%%%%%%%%%%%%%%%%%%%%%%
\section*{Exercise 3.10}

Show that for any orthonormal basis 
$B = \{\ket{\beta_1},\ket{\beta_2},\dots, \ket{\beta_n}\}$ 
and \\ vectors $\ket{v} = a_1\ket{\beta_1} + a_2\ket{\beta_2} + \dots + a_n\ket{\beta_n}$ 
and $\ket{w} = c_1\ket{\beta_1} + c_2\ket{\beta_2} + \dots + c_n\ket{\beta_n}$

\begin{itemize}
    \item[$\textbf{a.}$] the inner product $(w|v)$ of $\ket{v}$ and $\ket{w}$ is
        $\overline{c_1}a_1 + \overline{c_2}a_2 + \dots + \overline{c_n}a_n$ \\
        Since $\ket{w} = \begin{bmatrix}c_1 \\ c_2 \\ \vdots \\ c_n \end{bmatrix}$,
            we have 
            $\bra{w} = \begin{bmatrix}\overline{c_1} & \overline{c_2} & \cdots & \overline{c_n} \end{bmatrix}$. 
            Then taking the product $\bra{w}\ket{v}$ is just multiplying the row vector $\bra{w}$ with the column vector $\ket{v}$
            So we have 
            $$\begin{bmatrix}\overline{c_1} & \overline{c_2} & \cdots & \overline{c_n} 
            \end{bmatrix}
            \begin{bmatrix}
                a_1 \\ a_2 \\ \vdots \\ a_n
            \end{bmatrix}$$
        And the result is trivially $\overline{c_1}a_1 + \overline{c_2}a_2 + \dots + \overline{c_n}a_n$
   % \begin{align*}
   % \end{align*}

\item[$\textbf{b.}$] the length squared of $\ket{v}$ is
    $|\ket{v}|^2 = \bra{v}\ket{v} = |a_1|^2 + |a_2|^2 + \dots + |a_n|^2$.
    \begin{align*}
        \bra{v}\ket{v} &= \bra{v}\ket{ a_1\ket{\beta_1} + a_2\ket{\beta_2} + \dots + a_n\ket{\beta_n}} \\
        &= a_1\bra{v}\ket{\beta_1} + a_2\bra{v}\ket{\beta_2} + \dots + a_n\bra{v}\ket{\beta_n} \\
        &= a_1\overline{\bra{\beta_1}\ket{v}} + a_2\overline{\bra{\beta_2}\ket{v}} + \dots + a_n\overline{\bra{\beta_n}\ket{v}} \\
        &= a_1\overline{\bra{\beta_1}\ket{a_1\ket{\beta_1} + a_2\ket{\beta_2} + \dots + a_n\ket{\beta_n}}} + \dots + a_n\overline{\bra{\beta_n}\ket{
a_1\ket{\beta_1} + a_2\ket{\beta_2} + \dots + a_n\ket{\beta_n} }} \\
&= a_1\overline{a_1}\bra{\beta_1}\ket{\beta_1} + a_1\overline{a_2}\bra{\beta_1}\ket{\beta_2} + \dots + a_n\overline{a_1}\bra{\beta_n}\ket{\beta_1} + \dots + a_n\overline{a_n}\bra{\beta_n}\ket{\beta_n} \\
        &= a_1\overline{a_1} + \dots + a_n\overline{a_n} \ \ \ \ \ \   (\text{Orthonormality}) \\
        &= |a_1|^2 + \dots + |a_n|^2
    \end{align*}
\end{itemize}
Write all the steps in Dirac's bra/ket notation.
%%%%%%%%%%%%%%%%%%%%%%%%%%%%%%%%%%%%%%%%%%%%%%%%%%%%%%%%%%%%
\section*{Exercise 3.12}

Give an example of a two-qubit state that is a superposition with respect
to the standard basis but that is not entangled.

Consider the following state:
$$\ket{00} + \ket{01} + \ket{10} + \ket{11}$$
This is a proper linear combination of the basis vectors, so it is a superposition.
And it is obviously not entangled with respect to the decomposition into two single qubits.
E.g., the equation $$(c_0\ket{0} + d_0\ket{1}) \otimes (c_1\ket{0} + d_1\ket{1}) = c_0c_1\ket{00} + 
c_0d_1\ket{01} + d_0c_1\ket{10} + d_0d_1\ket{11}$$
holds with each coefficient equal to 1. So this is a superposition but not entangled.

%%%%%%%%%%%%%%%%%%%%%%%%%%%%%%%%%%%%%%%%%%%%%%%%%%%%%%%%%%%%
\section*{Exercise 3.13}

\begin{itemize}
    \item[$\textbf{a.}$] Show that the four-qubit state 
        $\ket{\psi} = \frac{1}{2}(\ket{00} + \ket{11} + \ket{22} + \ket{33})$ of 
        example 3.2.3 is entangled with respect to the decomposition into two 
        two-qubit subsytems consisting of the first and second qubits 
        and the third and fourth qubits.
    \begin{proof}
        We show $\ket{\psi}$ is entangled with respect to the two two-qubit 
        decomposition. To do this, we first assume that it is unentangled with respect
        to this decomposition. That is, $\ket{\psi} = \frac{1}{2}(\ket{0000} + \ket{0101} + \ket{1010} + \ket{1111})$
        unentangled with respect to the two two-qubit decomposition.
        That is, we assume $\ket{\psi} = (c_0\ket{0}_0\ket{0}_1 + d_0\ket{1}_0\ket{1}_1) 
        \otimes (c_1\ket{0}_2\ket{0}_3 + d_1\ket{1}_2\ket{1}_3)$
        Carrying out the tensor product gives us:
        $$\ket{\psi} = c_0c_1\ket{0000} + d_0c_1\ket{1100} + c_0d_1\ket{0011} + d_0d_1\ket{1111}$$
        So we have the following equations 
        \begin{align*}
            c_0c_1 &= \haf & d_0c_1  = 0 \\
            c_0d_1 &= 0 & d_0d_1 = \haf 
        \end{align*}
        Since $c_0d_1 = 0$, and we are in an integral domain, we must have either $c_0 = 0$ or $d_1 = 0$.
        In the case that $c_0 = 0$, we have a contradiction because $c_0c_1 = \haf$.
        In the case that $d_1 = 0$, we have a contradiction because $d_0d_1 = \haf$. \\
        Therefore our assumption that $\ket{\psi}$ was unentangled with respect to this decomposition was wrong,
        and we conclude that $\ket{\psi}$ is entangled with respect to the decomposition into two two-qubit subsystems consisting of the first and second qubits and the third and fourth qubits.
    \end{proof}
    \item[$\textbf{b.}$] For the four decompositions into two subsytems consisting 
        of one and three qubits, say whether $\ket{\psi}$ is entangled or unentangled
        with respect to each of these decompositions. \\
        We will do each decomposition and split the cases based on which qubit is the single qubit (i.e. not in the three qubits).
        Recall the definition of $\ket{\psi} = \haf(\ket{0000} + \ket{0101} + \ket{1010} + \ket{1111}) $.
    \begin{enumerate}
        \item[0.] $\ket{\psi} = (c_0\ket{0}_0 + d_0\ket{1}_0) \otimes (c_1\ket{0}_1\ket{0}_2\ket{0}_3 + d_1\ket{1}_1\ket{1}_2\ket{1}_3)$ \\
            Assume that $\ket{\psi}$ is unentangled with respect to this decomposition.
            \begin{align*} 
                \ket{\psi} 
                &= (c_0\ket{0}_0 + d_0\ket{1}_0) \otimes (c_1\ket{0}_1\ket{0}_2\ket{0}_3 + d_1\ket{1}_1\ket{1}_2\ket{1}_3) \\
                &= c_0c_1\ket{0000} + c_0d_1\ket{0111}  + d_0c_1\ket{1000} + d_0d_1\ket{1111}
            \end{align*}
            So we must have
            \begin{align*}
                c_0c_1 &= \haf & c_0d_1 = 0 \\
                d_0c_1 &= 0 & d_0d_1 = \haf 
            \end{align*}
            Since we are an integral domain and $c_0d_1 = 0$, we must have either $c_0 = 0$ or $d_0 = 0$. 
            But we also have $c_0c_1 = \haf$ and $d_0d_1 = \haf$, so either way we have a contradiction.
            Therefore $\ket{\psi}$ is entangled wtih respect to the decomposition into the first qubit and the remaining qubits.
            \newpage
        \item[1.]
        $\ket{\psi} = (c_0\ket{0}_1 + d_0\ket{1}_1) 
        \otimes (c_1\ket{0}_0\ket{0}_2\ket{0}_3 + d_1\ket{1}_0\ket{1}_2\ket{1}_3)$ \\
            We have a map $\phi : V_0 \otimes V_1 \otimes V_2 \otimes V_3 \simeq V_1 \otimes (V_0 \otimes V_2 \otimes V_3)$ \\
            Defined by $\phi(v_0 \otimes v_1 \otimes v_2 \otimes v_3) = v_1 \otimes (v_0 \otimes v_2 \otimes v_3) $ \\
            So we will need to consider $\phi(\ket{\psi})$, which is equal to 
            $$\haf(\ket{0}_1\ket{0}_0\ket{0}_2\ket{0}_3 + \ket{1}_1\ket{0}_0\ket{0}_2\ket{1}_3 + \ket{0}_1\ket{1}_0\ket{1}_2\ket{0}_3 + \ket{1}_1 \ket{1}_0\ket{1}_2\ket{1}_3)$$
            Assume that $\phi(\ket{\psi})$ is unentangled with respect to this decomposition.
            \begin{align*}
                \phi(\ket{\psi})
                &= (c_0\ket{0}_1 + d_0\ket{1}_1) \otimes (d_1\ket{0}_0\ket{0}_2\ket{0}_3 + \cdots + d_8\ket{1}_0\ket{1}_2\ket{1}_3) \\
                &= c_0d_1\ket{0}_1\ket{0}_0\ket{0}_2\ket{0}_3 + d_0d_1\ket{1}_1\ket{0}_0\ket{0}_2\ket{0}_3 + \cdots + d_0d_8\ket{1}_1\ket{1}_0\ket{1}_2\ket{1}_3
            \end{align*}
            This gives us enough equations to work with. 
            So we must have: 
            \begin{align*}
                c_0d_1 &= \haf \\
                d_0d_1 &= 0 \\
                d_0d_8 &= \haf
            \end{align*}
            The second equation means that either $d_0 = 0$ or $d_1 = 0$. Since $c_0d_1 = \haf$, we must have $d_0 = 0$.
            Yet, $d_0d_8 = \haf$, so there is a contradiction. 
            Hence $\phi(\ket{\psi})$, and therefore $\ket{\psi}$, is entangled with respect to this decomposition.
        %%%%%%%%%%%%%%%%%%%%%%%%%%%%%%%%%%%%%%%%%%%%%%%%%%
            \newpage
        \item[2.]
        $\ket{\psi} = (c_0\ket{0}_2 + d_0\ket{1}_2) 
        \otimes (c_1\ket{0}_0\ket{0}_1\ket{0}_3 + d_1\ket{1}_0\ket{1}_1\ket{1}_3)$ \\
        We have a map $\phi : V_0 \otimes V_1 \otimes V_2 \otimes V_3 \simeq V_2 \otimes (V_0 \otimes V_1 \otimes V_3)$ \\
            Defined by $\phi(v_0 \otimes v_1 \otimes v_2 \otimes v_3) = v_2 \otimes (v_0 \otimes v_1 \otimes v_3) $ \\
            So we will need to consider $\phi(\ket{\psi})$, which is equal to 
            $$\haf(\ket{0}_2\ket{0}_0\ket{0}_1\ket{0}_3 + \ket{1}_2\ket{0}_0\ket{0}_1\ket{1}_3 + \ket{0}_2\ket{1}_0\ket{1}_1\ket{0}_3 + \ket{1}_2 \ket{1}_0\ket{1}_1\ket{1}_3)$$
            Assume that $\phi(\ket{\psi})$ is unentangled with respect to this decomposition.
            \begin{align*}
                \phi(\ket{\psi})
                &= (c_0\ket{0}_2 + d_0\ket{1}_2) \otimes (d_1\ket{0}_0\ket{0}_1\ket{0}_3 + \cdots + d_8\ket{1}_0\ket{1}_1\ket{1}_3) \\
                &= c_0d_1\ket{0}_2\ket{0}_0\ket{0}_1\ket{0}_3 + d_0d_1\ket{1}_2\ket{0}_0\ket{0}_1\ket{0}_3 + \cdots + d_0d_8\ket{1}_2\ket{1}_0\ket{1}_1\ket{1}_3
            \end{align*}
            We can study the coefficients of the first two terms and the last term (those written above).
            \begin{align*}
                c_0d_1 &= \haf \\
                d_0d_1 &= 0 \\
                d_0d_8 &= \haf
            \end{align*}
         The second equation means that either $d_0 = 0$ or $d_1 = 0$. Since $c_0d_1 = \haf$, we must have $d_0 = 0$.
            Yet, $d_0d_8 = \haf$, so there is a contradiction. 
            Hence $\phi(\ket{\psi})$, and therefore $\ket{\psi}$, is entangled with respect to this decomposition.

            
        %%%%%%%%%%%%%%%%%%%%%%%%%%%%%%%%%%%%%%%%%%%%%%%%%%
        \newpage
        \item[3.]
        $\ket{\psi} = (c_0\ket{0}_3 + d_0\ket{1}_3) 
        \otimes (c_1\ket{0}_0\ket{0}_1\ket{0}_2 + d_1\ket{1}_0\ket{1}_1\ket{1}_2)$

             We have a map $\phi : V_0 \otimes V_1 \otimes V_2 \otimes V_3 \simeq V_3 \otimes (V_0 \otimes V_1 \otimes V_2)$ \\
            Defined by $\phi(v_0 \otimes v_1 \otimes v_2 \otimes v_3) = v_3 \otimes (v_0 \otimes v_1 \otimes v_2) $ \\
            So we will need to consider $\phi(\ket{\psi})$, which is equal to 
            $$\haf(\ket{0}_3\ket{0}_0\ket{0}_1\ket{0}_2 + \ket{1}_3\ket{0}_0\ket{0}_1\ket{1}_2 + \ket{0}_3\ket{1}_0\ket{1}_1\ket{0}_2 + \ket{1}_3 \ket{1}_0\ket{1}_1\ket{1}_2)$$
            Assume that $\phi(\ket{\psi})$ is unentangled with respect to this decomposition.
            \begin{align*}
                \phi(\ket{\psi})
                &= (c_0\ket{0}_3 + d_0\ket{1}_3) \otimes (d_1\ket{0}_0\ket{0}_1\ket{0}_2 + \cdots + d_8\ket{1}_0\ket{1}_1\ket{1}_2) \\
                &= c_0d_1\ket{0}_3\ket{0}_0\ket{0}_1\ket{0}_2 + d_0d_1\ket{1}_3\ket{0}_0\ket{0}_1\ket{0}_2 + \cdots + d_0d_8\ket{1}_3\ket{1}_0\ket{1}_1\ket{1}_2
            \end{align*}
            We can study the coefficients of the first two terms and the last term (those written above).
            \begin{align*}
                c_0d_1 &= \haf \\
                d_0d_1 &= 0 \\
                d_0d_8 &= \haf
            \end{align*}
         The second equation means that either $d_0 = 0$ or $d_1 = 0$. Since $c_0d_1 = \haf$, we must have $d_0 = 0$.
            Yet, $d_0d_8 = \haf$, so there is a contradiction. 
            Hence $\phi(\ket{\psi})$, and therefore $\ket{\psi}$, is entangled with respect to this decomposition.
        \end{enumerate}
\end{itemize}

%%%%%%%%%%%%%%%%%%%%%%%%%%%%%%%%%%%%%%%%%%%%%%%%%%%%%%%%%%%%
\newpage
\section*{Exercise 4.1}

Give the matrix, in the standard basis, for the following operators

\begin{enumerate}
    \item[$\textbf{a.}$] $\ket{0}\bra{0}$  \\
        \begin{align*}
            \ket{0}\bra{0} &= \begin{bmatrix}1 \\ 0 \end{bmatrix} \begin{bmatrix}1 & 0 \end{bmatrix} \\
                &= \begin{bmatrix}1 & 0 \\ 0 & 0 \end{bmatrix} 
        \end{align*}
    \item[$\textbf{b.}$] $\ket{+}\bra{0} - \bfi\ket{-}\bra{1}$
        \begin{align*}
            \ket{+}\bra{0}  
                &= 
                 \begin{bmatrix}\srtt \\ \srtt \end{bmatrix}
                 \begin{bmatrix}1 & 0\end{bmatrix}  \\
                     &= \begin{bmatrix} \srtt & 0 \\ \srtt & 0 \end{bmatrix}
                         & 
        \end{align*}
        \begin{align*}
            \bfi\ket{-}\bra{1} 
                &= \bfi(\begin{bmatrix}\srtt \\ -\srtt \end{bmatrix}
                    \begin{bmatrix} 0  & 1 \end{bmatrix})  \\
                &= \bfi\begin{bmatrix}
                    0 & \srtt \\
                    0 & -\srtt 
                \end{bmatrix} \\
                &= \begin{bmatrix}
                    0 & \frac{\bfi}{\sqrt{2}} \\
                    0 & -\frac{\bfi}{\sqrt{2}} 
                \end{bmatrix}  & 
        \end{align*}
        \begin{align*}
            \ket{+}\bra{0} - \bfi\ket{-}\bra{1} 
            &= \begin{bmatrix} \srtt & 0 \\ \srtt & 0 \end{bmatrix}
                - \begin{bmatrix}
                    0 & \frac{\bfi}{\sqrt{2}} \\
                    0 & -\frac{\bfi}{\sqrt{2}} 
                \end{bmatrix} \\
                &= \begin{bmatrix}
                    \srtt & -\frac{\bfi}{\sqrt{2}} \\
                    \srtt & \frac{\bfi}{\sqrt{2}}
                \end{bmatrix} & 
        \end{align*}
    \item[$\textbf{c.}$] $\ket{00}\bra{00} + \ket{01}\bra{01}$
        \begin{align*}
        \ket{00}\bra{00} + \ket{01}\bra{01} 
            &= \begin{bmatrix}
                1 \\ 0 \\ 0 \\ 0
                \end{bmatrix}
                \begin{bmatrix}
                    1 & 0 & 0 & 0
                \end{bmatrix} + 
                \begin{bmatrix}
                    0 \\ 1 \\ 0 \\ 0
                \end{bmatrix}
                \begin{bmatrix}
                    0 & 1 & 0 & 0
                \end{bmatrix} \\
            &= \begin{bmatrix}
                1 & 0 & 0 & 0 \\
                0 & 0 & 0 & 0 \\
                0 & 0 & 0 & 0 \\
                0 & 0 & 0 & 0 \\
         \end{bmatrix}
            + 
            \begin{bmatrix}
                0 & 0 & 0 & 0 \\
                0 & 1 & 0 & 0 \\
                0 & 0 & 0 & 0 \\
                0 & 0 & 0 & 0 \\
            \end{bmatrix} \\
        &= \begin{bmatrix}
                1 & 0 & 0 & 0 \\
                0 & 1 & 0 & 0 \\
                0 & 0 & 0 & 0 \\
                0 & 0 & 0 & 0 \\
            \end{bmatrix} 
        \end{align*}
    \item[$\textbf{d.}$] $\ket{00}\bra{00} + \ket{01}\bra{01} + \ket{11}\bra{01} + 
        \ket{10}\bra{11}$ \\
        Here we can use the result we just found to simplify this expression. 
        We know  $$\ket{00}\bra{00} + \ket{01}\bra{01}
        = \begin{bmatrix}
                1 & 0 & 0 & 0 \\
                0 & 1 & 0 & 0 \\
                0 & 0 & 0 & 0 \\
                0 & 0 & 0 & 0 \\
            \end{bmatrix}$$
        So we shall find  $\ket{11}\bra{01} + \ket{10}\bra{11}$, 
        and then add the two matrices.
        \begin{align*}
        \ket{11}\bra{01} + \ket{10}\bra{11}
            &= \begin{bmatrix}
                0 \\ 0 \\ 0 \\ 1
                \end{bmatrix}
                \begin{bmatrix}
                    0 & 1 & 0 & 0
                \end{bmatrix} + 
                \begin{bmatrix}
                    0 \\ 0 \\ 1 \\ 0
                \end{bmatrix}
                \begin{bmatrix}
                    0 & 0 & 0 & 1
                \end{bmatrix} \\
            &= \begin{bmatrix}
                0 & 0 & 0 & 0 \\
                0 & 0 & 0 & 0 \\
                0 & 0 & 0 & 0 \\
                0 & 1 & 0 & 0 \\
         \end{bmatrix}
            + 
            \begin{bmatrix}
                0 & 0 & 0 & 0 \\
                0 & 0 & 0 & 0 \\
                0 & 0 & 0 & 1 \\
                0 & 0 & 0 & 0 \\
            \end{bmatrix} \\
        &= \begin{bmatrix}
                0 & 0 & 0 & 0 \\
                0 & 0 & 0 & 0 \\
                0 & 0 & 0 & 1 \\
                0 & 1 & 0 & 0 \\
            \end{bmatrix}
        \end{align*}
    Now we can combine the results:
    \begin{align*}
        \ket{00}\bra{00} + \ket{01}\bra{01} + \ket{11}\bra{01} + \ket{10}\bra{11}
        &= \begin{bmatrix}
                1 & 0 & 0 & 0 \\
                0 & 1 & 0 & 0 \\
                0 & 0 & 0 & 0 \\
                0 & 0 & 0 & 0 \\
            \end{bmatrix} + 
            \begin{bmatrix}
                0 & 0 & 0 & 0 \\
                0 & 0 & 0 & 0 \\
                0 & 0 & 0 & 1 \\
                0 & 1 & 0 & 0 \\
            \end{bmatrix} \\
        &= \begin{bmatrix}
                1 & 0 & 0 & 0 \\
                0 & 1 & 0 & 0 \\
                0 & 0 & 0 & 1 \\
                0 & 1 & 0 & 0 \\
            \end{bmatrix}
    \end{align*}

    \item[$\textbf{e.}$] $\ket{\Psi^+}\bra{\Psi^+}$ where 
        $\ket{\Psi^+} = \srtt(\ket{00} + \ket{11})$
    \begin{align*}
        \srtt(\ket{00} + \ket{11}) 
        &= \srtt(\begin{bmatrix}1\\0\\0\\0\end{bmatrix}+\begin{bmatrix}0\\0\\0\\1\end{bmatrix}) \\
        &= \srtt\begin{bmatrix}1\\0\\0\\1\end{bmatrix}
    \end{align*}

    \begin{align*}
        \bsp\bra{\Psi^+} &= \srtt\begin{bmatrix}1\\0\\0\\1\end{bmatrix}
            \srtt\begin{bmatrix}1&0&0&1\end{bmatrix} \\
            &= \haf \begin{bmatrix}1\\0\\0\\1\end{bmatrix}
            \begin{bmatrix}1&0&0&1\end{bmatrix} \\
            &= \haf \begin{bmatrix}
                1 & 0 & 0 & 1 \\
                0 & 0 & 0 & 0 \\
                0 & 0 & 0 & 0 \\
                1 & 0 & 0 & 1 \\
            \end{bmatrix}
    \end{align*}

\end{enumerate}


%%%%%%%%%%%%%%%%%%%%%%%%%%%%%%%%%%%%%%%%%%%%%%%%%%%%%%%%%%%%
\newpage
\section*{Exercise 4.2} Write the following operators in bra/ket notation

\begin{enumerate}
    \item[$\textbf{a.}$]
    The Hadamard operator 
        $H = \begin{pmatrix}\srtt & \srtt \\ \srtt & -\srtt \end{pmatrix}$
    \\ 
    From the previous exercise, we know 
        $$\ket{+}\bra{0} = \begin{bmatrix} \srtt & 0 \\ \srtt & 0 \end{bmatrix}$$
    Consider $\ket{-}\bra{1}$. 
    \begin{align*}
        \ket{-}\bra{1} &= \begin{bmatrix}\srtt \\ -\srtt\end{bmatrix}
            \begin{bmatrix}0 & 1\end{bmatrix} \\
            &= \begin{bmatrix}
                0 & \srtt \\
                0 & -\srtt
            \end{bmatrix}
    \end{align*}
    We can get the desired matrix by adding these matrices:
        \begin{align*}
        \ket{+}\bra{0} + \ket{-}\bra{1}
            &= \begin{bmatrix} 
                \srtt & 0 \\ 
                \srtt & 0 
                \end{bmatrix}
                + 
                \begin{bmatrix}
                0 & \srtt \\
                0 & -\srtt
                \end{bmatrix} \\
            &= \begin{bmatrix}
                \srtt & \srtt \\
                \srtt & -\srtt
                \end{bmatrix}
        \end{align*}
    So we know the bra/ket notation for the Hadamard operation is: 
        $\ket{+}\bra{0} + \ket{-}\bra{1}$
\newpage
    \item[$\textbf{b.}$]
    $X = \begin{pmatrix}0 & 1 \\ 1 & 0 \end{pmatrix}$ \\
        The notation is $\ket{1}\bra{0} + \ket{0}\bra{1}$.
\begin{align*}
        \ket{1}\bra{0} + \ket{0}\bra{1}
        &= \begin{bmatrix}
            0 \\ 1
        \end{bmatrix}
        \begin{bmatrix}
            1 & 0
        \end{bmatrix}
        + \ket{0}\bra{1} \\
        &= \begin{bmatrix}
            0 \\ 1
        \end{bmatrix}
        \begin{bmatrix}
            1 & 0
        \end{bmatrix}
        +
        \begin{bmatrix}
            1 \\ 0
        \end{bmatrix}
        \begin{bmatrix}
            0 & 1
        \end{bmatrix} \\
        &= \begin{bmatrix}
            0 & 0 \\
            1 & 0 
        \end{bmatrix}
        + 
        \begin{bmatrix}
            0 & 1 \\
            0 & 0 
        \end{bmatrix} \\
        &= \begin{bmatrix}
            0 & 1 \\
            1 & 0
        \end{bmatrix}
    \end{align*}

    \item[$\textbf{c.}$]
    $Y = \begin{pmatrix}0 & 1 \\ -1 & 0 \end{pmatrix}$.
        The notation is $\ket{0}\bra{1} - \ket{1}\bra{0}$. 
        We know the matrices for the two terms.
        \begin{align*}
            \ket{0}\bra{1} &= \begin{bmatrix}
            0 & 1 \\
            0 & 0 
        \end{bmatrix} \\
            \ket{1}\bra{0} &= \begin{bmatrix}
            0 & 0 \\
            1 & 0 
        \end{bmatrix} \\
            \ket{0}\bra{1} - \ket{1}\bra{0}
            &= \begin{bmatrix}
            0 & 1 \\
            0 & 0 
        \end{bmatrix} -
        \begin{bmatrix}
            0 & 0 \\
            1 & 0 
        \end{bmatrix} \\
            &= 
        \begin{bmatrix}
            0 & 1 \\
            -1 & 0 
        \end{bmatrix}
        \end{align*}

    \item[$\textbf{d.}$]
    $Z = \begin{pmatrix}1 & 0 \\ 0 & -1 \end{pmatrix}$.
 The bra/ket notation is $\ket{0}\bra{0} - \ket{1}\bra{1}$.
    We verify this now: 
    \begin{align*}
    \ket{0}\bra{0} - \ket{1}\bra{1} 
        &= \begin{bmatrix}
            1 \\ 0
        \end{bmatrix}
        \begin{bmatrix}
            1 & 0
        \end{bmatrix}
        - \ket{1}\bra{1} \\
        &= \begin{bmatrix}
            1 \\ 0
        \end{bmatrix}
        \begin{bmatrix}
            1 & 0
        \end{bmatrix}
        - 
        \begin{bmatrix}
            0 \\ 1
        \end{bmatrix}
        \begin{bmatrix}
            0 & 1
        \end{bmatrix} \\
        &= \begin{bmatrix}
            1 & 0 \\
            0 & 0 
        \end{bmatrix}
        - 
        \begin{bmatrix}
            0 & 0 \\
            0 & 1 
        \end{bmatrix} \\
        &= \begin{bmatrix}
            1 & 0 \\
            0 & -1
        \end{bmatrix}
    \end{align*}


    \item[$\textbf{e.}$]
        $\begin{pmatrix}
            23 & 0 & 0 & 0 \\ 
            0 & -5 & 0 & 0 \\
            0 & 0 & 0 & 0 \\
            0 & 0 & 0 & 9
        \end{pmatrix}$
    We claim the notation is: 
    $23\ket{00}\bra{00} - 5\ket{01}\bra{01} + 9\ket{11}\bra{11}$. 
    \begin{align*}
        \ket{00}\bra{00} &= 
        \begin{bmatrix}
            1 & 0 & 0 & 0 \\
            0 & 0 & 0 & 0 \\
            0 & 0 & 0 & 0 \\
            0 & 0 & 0 & 0 \\
        \end{bmatrix}\\
        23\ket{00}\bra{00} &= 
        \begin{bmatrix}
            23 & 0 & 0 & 0 \\
            0 & 0 & 0 & 0 \\
            0 & 0 & 0 & 0 \\
            0 & 0 & 0 & 0 \\
        \end{bmatrix}\\
    \end{align*}
    \begin{align*}
        \ket{01}\bra{01} &= 
        \begin{bmatrix}
        0 \\ 1 \\ 0 \\ 0
        \end{bmatrix}
        \begin{bmatrix}
            0 & 1 & 0 & 0
        \end{bmatrix} \\
        &= \begin{bmatrix}
            0 & 0 & 0 & 0 \\
            0 & 1 & 0 & 0 \\
            0 & 0 & 0 & 0 \\
            0 & 0 & 0 & 0 \\
        \end{bmatrix} \\
        5\ket{01}\bra{01} &= \begin{bmatrix}
            0 & 0 & 0 & 0 \\
            0 & 5 & 0 & 0 \\
            0 & 0 & 0 & 0 \\
            0 & 0 & 0 & 0 \\
        \end{bmatrix} \\
    \end{align*}
    \begin{align*}
        \ket{11}\bra{11} &= 
        \begin{bmatrix}
            0 & 0 & 0 & 0 \\
            0 & 0 & 0 & 0 \\
            0 & 0 & 0 & 0 \\
            0 & 0 & 0 & 1 \\
        \end{bmatrix} \\
        9\ket{11}\bra{11} &= \begin{bmatrix}
            0 & 0 & 0 & 0 \\
            0 & 0 & 0 & 0 \\
            0 & 0 & 0 & 0 \\
            0 & 0 & 0 & 9 \\
        \end{bmatrix}
    \end{align*}
    \begin{align*}
        23\ket{00}\bra{00} - 5\ket{01}\bra{01} + 9\ket{11}\bra{11}
        &= \begin{bmatrix}
            23 & 0 & 0 & 0 \\
            0 & 0 & 0 & 0 \\
            0 & 0 & 0 & 0 \\
            0 & 0 & 0 & 0 \\
        \end{bmatrix} - \begin{bmatrix}
            0 & 0 & 0 & 0 \\
            0 & 5 & 0 & 0 \\
            0 & 0 & 0 & 0 \\
            0 & 0 & 0 & 0 \\
        \end{bmatrix}  + 
        \begin{bmatrix}
            0 & 0 & 0 & 0 \\
            0 & 0 & 0 & 0 \\
            0 & 0 & 0 & 0 \\
            0 & 0 & 0 & 9 \\
        \end{bmatrix} \\
        &= \begin{bmatrix}
            23 & 0 & 0 & 0 \\ 
            0 & -5 & 0 & 0 \\
            0 & 0 & 0 & 0 \\
            0 & 0 & 0 & 9
        \end{bmatrix}
    \end{align*}
\newpage
\item[$\textbf{f.}$] $X \otimes X$.
    $$\text{The notation is: } \ket{00}\bra{11} + \ket{01}\bra{10} + \ket{10}\bra{01}
    + \ket{11}\bra{00}$$
    \begin{align*}
        X \otimes X &= 
                    \begin{bmatrix}0 & 1 \\ 1 & 0 \end{bmatrix}
                        \otimes
                    \begin{bmatrix}0 & 1 \\ 1 & 0 \end{bmatrix} \\
                    &= 
                    \begin{bmatrix}
                        0 \cdot \begin{pmatrix}0 & 1 \\ 1 & 0 \end{pmatrix} 
                    &   1 \cdot \begin{pmatrix}0 & 1 \\ 1 & 0 \end{pmatrix}  \\
                        1 \cdot \begin{pmatrix}0 & 1 \\ 1 & 0 \end{pmatrix} 
                    &   0 \cdot \begin{pmatrix}0 & 1 \\ 1 & 0 \end{pmatrix}  \\
                    \end{bmatrix} \\
                    &= \begin{bmatrix}
                    0 & 0 & 0 & 1\\
                    0 & 0 & 1 & 0\\
                    0 & 1 & 0 & 0\\
                    1 & 0 & 0 & 0\\
                    \end{bmatrix} \\
                    &= 
                    \begin{bmatrix}
                    0 & 0 & 0 & 1\\
                    0 & 0 & 0 & 0\\
                    0 & 0 & 0 & 0\\
                    0 & 0 & 0 & 0\\
                    \end{bmatrix} +
                    \begin{bmatrix}
                    0 & 0 & 0 & 0\\
                    0 & 0 & 1 & 0\\
                    0 & 0 & 0 & 0\\
                    0 & 0 & 0 & 0\\
                    \end{bmatrix} +
                    \begin{bmatrix}
                    0 & 0 & 0 & 0\\
                    0 & 0 & 0 & 0\\
                    0 & 1 & 0 & 0\\
                    0 & 0 & 0 & 0\\
                    \end{bmatrix} + 
                    \begin{bmatrix}
                    0 & 0 & 0 & 0\\
                    0 & 0 & 0 & 0\\
                    0 & 0 & 0 & 0\\
                    1 & 0 & 0 & 0\\
                    \end{bmatrix} \\
                    &= \begin{bmatrix}1 \\ 0 \\ 0 \\ 0\end{bmatrix}
                       \begin{bmatrix}0 & 0 & 0 & 1\end{bmatrix} +
                       \begin{bmatrix}0 \\ 1 \\ 0 \\ 0\end{bmatrix}
                       \begin{bmatrix}0 & 0 & 1 & 0\end{bmatrix} +
                       \begin{bmatrix}0 \\ 0 \\ 1 \\ 0\end{bmatrix}
                       \begin{bmatrix}0 & 1 & 0 & 0\end{bmatrix} +
                       \begin{bmatrix}0 \\ 0 \\ 0 \\ 1\end{bmatrix}
                       \begin{bmatrix}1 & 0 & 0 & 0\end{bmatrix} \\
                    &= \ket{00}\bra{11} + \ket{01}\bra{10} + \ket{10}\bra{01} + \ket{11}\bra{00} 
    \end{align*}
\newpage
    \item[$\textbf{g.}$] $X \otimes Z$
    $$\text{The notation is: } 
    \ket{00}\bra{10} - \ket{01}\bra{11} + \ket{10}\bra{00} - \ket{11}\bra{01} $$
    \begin{align*}
        X \otimes Z &= 
                    \begin{bmatrix}0 & 1 \\ 1 & 0 \end{bmatrix}
                        \otimes
                    \begin{bmatrix}1 & 0 \\ 0 & -1 \end{bmatrix} \\
                    &= 
                    \begin{bmatrix}
                        0 \cdot \begin{pmatrix}1 & 0 \\ 0 & -1 \end{pmatrix} 
                    &   1 \cdot \begin{pmatrix}1 & 0 \\ 0 & -1 \end{pmatrix}  \\
                        1 \cdot \begin{pmatrix}1 & 0 \\ 0 & -1 \end{pmatrix} 
                    &   0 \cdot \begin{pmatrix}1 & 0 \\ 0 & -1 \end{pmatrix}  \\
                    \end{bmatrix} \\
                    &= \begin{bmatrix}
                    0 & 0 & 1 & 0\\
                    0 & 0 & 0 & -1\\
                    1 & 0 & 0 & 0\\
                    0 & -1 & 0 & 0\\
                    \end{bmatrix} \\
                    &= 
                    \begin{bmatrix}
                    0 & 0 & 1 & 0\\
                    0 & 0 & 0 & 0\\
                    0 & 0 & 0 & 0\\
                    0 & 0 & 0 & 0\\
                    \end{bmatrix} -
                    \begin{bmatrix}
                    0 & 0 & 0 & 0\\
                    0 & 0 & 0 & 1\\
                    0 & 0 & 0 & 0\\
                    0 & 0 & 0 & 0\\
                    \end{bmatrix} +
                    \begin{bmatrix}
                    0 & 0 & 0 & 0\\
                    0 & 0 & 0 & 0\\
                    1 & 0 & 0 & 0\\
                    0 & 0 & 0 & 0\\
                    \end{bmatrix} -
                    \begin{bmatrix}
                    0 & 0 & 0 & 0\\
                    0 & 0 & 0 & 0\\
                    0 & 0 & 0 & 0\\
                    0 & 1 & 0 & 0\\
                    \end{bmatrix} \\
                    &= \begin{bmatrix}1 \\ 0 \\ 0 \\ 0\end{bmatrix}
                       \begin{bmatrix}0 & 0 & 1 & 0\end{bmatrix} -
                       \begin{bmatrix}0 \\ 1 \\ 0 \\ 0\end{bmatrix}
                       \begin{bmatrix}0 & 0 & 0 & 1\end{bmatrix} +
                       \begin{bmatrix}0 \\ 0 \\ 1 \\ 0\end{bmatrix}
                       \begin{bmatrix}1 & 0 & 0 & 0\end{bmatrix} -
                       \begin{bmatrix}0 \\ 0 \\ 0 \\ 1\end{bmatrix}
                       \begin{bmatrix}0 & 1 & 0 & 0\end{bmatrix} \\
                    &= \ket{00}\bra{10} - \ket{01}\bra{11} + \ket{10}\bra{00} - \ket{11}\bra{01} 
    \end{align*}
\newpage
    \item[$\textbf{h.}$] $H \otimes H$ 
        We can configure the matrix by adding the action of $H \otimes H$ on each basis vector of the two-qubit state space basis.
        That is, the notation follows from $$(H \otimes H)(\ket{0} \otimes \ket{0}) + (H \otimes H)(\ket{0} \otimes \ket{1}) + (H \otimes H)(\ket{1} \otimes \ket{0})
        + (H \otimes H)(\ket{1} \otimes \ket{1})$$
        We simplify this by using the properties of tensor products.
        \begin{align*}
            (H \otimes H)(\ket{0} \otimes \ket{0}) &= H\ket{0} \otimes H\ket{0} = \ket{+}\ket{+} \\
            (H \otimes H)(\ket{0} \otimes \ket{1}) &= H\ket{0} \otimes H\ket{1} = \ket{+}\ket{-} \\
            (H \otimes H)(\ket{1} \otimes \ket{0}) &= H\ket{1} \otimes H\ket{0} = \ket{-}\ket{+} \\
            (H \otimes H)(\ket{1} \otimes \ket{1}) &= H\ket{1} \otimes H\ket{1} = \ket{-}\ket{-} \\
        \end{align*}
        Therefore the Dirac notation is 
        $$ (\ket{+}\ket{+}\bra{0}\bra{0}) + (\ket{+}\ket{-}\bra{0}\bra{1}) + (\ket{-}\ket{+} \bra{1}\bra{0}) + (\ket{-}\ket{-}\bra{1}\bra{1}) $$

\end{enumerate}

%%%%%%%%%%%%%%%%%%%%%%%%%%%%%%%%%%%%%%%%%%%%%%%%%%%%%%%%%%%%


\end{document}
