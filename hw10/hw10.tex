\documentclass{article}
\usepackage[T1]{fontenc}
\usepackage[utf8]{inputenc}
\usepackage{verbatim}
\usepackage{mathrsfs}
\usepackage{url}
\usepackage{amsmath}
\usepackage{amsthm}
\usepackage{amssymb}
\usepackage{graphicx}
\usepackage{booktabs}
\usepackage{stmaryrd}
\usepackage{xcolor}

\usepackage{dsfont}
\usepackage{listings}
\usepackage{fancyhdr}
\pagestyle{fancy}
\usepackage{indentfirst}
\usepackage{layout}
\usepackage{hanging}
\usepackage{setspace}



\makeatletter
%%%%%%%%%%%%%%%%%%%%%%%%%%%%%% User specified LaTeX commands.
\usepackage[ backend=biber,sorting=none,maxbibnames=5]{biblatex}
\addbibresource{discreteGBKS.bib}
\usepackage{fullpage}
\usepackage{url}
\usepackage{bbm}
\usepackage{bbold}
\usepackage{amsthm}
\usepackage{multicol}
\usepackage{array}
\usepackage{mathrsfs}
\usepackage[all]{xy}
\usepackage{wesa}
\usepackage[braket,qm]{qcircuit}

\newcommand{\xsame}{\color{cyan}{\mathit{same}}}
\newcommand{\ysame}{\color{blue}{\mathit{same}}}
\newcommand{\zsame}{\color{red}{\mathit{same}}}

\newcommand{\xplus}{{\color{cyan}\rightarrow}}
\newcommand{\xminus}{{\color{cyan}\uparrow}}
\newcommand{\yplus}{{\color{blue}\nwarrow}}
\newcommand{\yminus}{{\color{blue}\swarrow}}
\newcommand{\zplus}{{\color{red}\sswarrow}}
\newcommand{\zminus}{{\color{red}\searrow}}

% \makeatletter
% \newtheoremstyle{indented}
%   {3pt}% space before
%   {3pt}% space after
%   {\addtolength{\@totalleftmargin}{3.5em}
%    \addtolength{\linewidth}{-3.5em}
%    \parshape 1 3.5em \linewidth}% body font
%   {}% indent
%   {\bfseries}% header font
%   {.}% punctuation
%   {.5em}% after theorem header
%   {}% header specification (empty for default)
% \makeatother
% \theoremstyle{indented}
\theoremstyle{remark}
\newtheorem{example}{Example}
\newtheorem{definition}{Definition}
\newtheorem{cor}{Corollary}
\newtheorem{lemma}{Lemma}
\newtheorem{thm}{Theorem}
\newtheorem{case}{Case}
\newtheorem{conjecture}{Conjecture}
\newtheorem{problem}{Problem}
\newtheorem{claim}{Claim}
\newcommand{\events}{\ensuremath{\mathcal{E}}}
\newcommand{\qevents}{\ensuremath{\mathcal{E}}}
\newcommand{\pmeas}{\ensuremath{\mu}}
\newcommand{\Hilb}{\mathcal{H}}
\newcommand{\ps}{\texttt{+}}
\newcommand{\ms}{\texttt{-}}
\newcommand{\poss}{{\mbox{\wesa{possible}}}}
\newcommand{\imposs}{{\mbox{\wesa{impossible}}}}
\newcommand{\likely}{{\mbox{\wesa{likely}}}}
\newcommand{\unlikely}{{\mbox{\wesa{unlikely}}}}
\newcommand{\necess}{{\mbox{\wesa{certain}}}}
\newcommand{\overflow}{{\mbox{\wesa{overflow}}}}
\newcommand{\unknown}{{\mbox{\wesa{unknown}}}}

\newcommand{\pzero}{\mathbb{0}}
\newcommand{\pone}{\mathbb{1}}
\def\C{{\mathbb{C}}}
\def\srtt{\frac{1}{\sqrt{2}}}
\newcommand{\ff}[1]{\mathbb{F}_{#1}}
\newcommand{\expect}[2]{ \langle #1 | #2 | #1 \rangle}
\newcommand{\Tr}{\mathop{\mathrm{Tr}}\nolimits}
\usepackage{tensor}
\newcommand{\muB}{\ensuremath{\mu^{B}}}
\newcommand{\rme}{\mathrm{e}}
\newcommand{\rmi}{\mathrm{i}}
\usepackage[braket]{qcircuit} 
\newcommand{\xyR}[1]{\xymatrixrowsep={#1}}
\newcommand{\xyC}[1]{\xymatrixcolsep={#1}}
%% \newcommand{\ip}[2]{\ensuremath{\left\langle{#1}\middle\vert{#2}\right\rangle}}
%% \newcommand{\melem}[3]{\ensuremath{\left\langle{#1}\middle\vert{#2}\middle\vert{#3}\right\rangle}}
%% \newcommand{\op}[2]{\ensuremath{\left\vert{#1}\middle\rangle\middle\langle{#2}\right\vert}}
\newcommand{\pr}[2]{\ip{#1}{#2}}
\newcommand{\proj}[1]{\op{#1}{#1}}
\usepackage{MnSymbol}
\newcommand{\HermitianForm}[2]{\ensuremath{\left\llangle{#1}\middle\vert{#2}\right\rrangle}}
\usepackage{csquotes}

\makeatother

\usepackage{babel}

\def\auths{Joshua Larkin}
\def\tit{Quantum Programming HW10}
\def\term{April 2020}
%\doublespacing

\lhead{\term}
\chead{\tit}
\rhead{\thepage}
\cfoot{}

\title{
    \vspace{2in}
    \textmd{\textbf{\tit}}\\
    \normalsize\vspace{0.1in}\small{B490 : Spring 2020 }\\
    \vspace{0.1in}\large{\textit{\auths}}
    \vspace{3in}
}

\date{}

\begin{document}

\maketitle
\pagebreak


%%%%%%%%%%%%%%%%%%%%%%%%%%%%%%%%%%%%%%%%%%%%%%%%%%%%%%%%%%%%%%%%%%%%%%%%
\section{Calculating Probabilities}
We know the state of the quantum system: $\ket{\phi} = \srtt(\ket{00} + \ket{11})$.
We show the probabilities of both detectors flashing the same color for all 
9 experiments.
\paragraph*{Experiment $\xplus\xplus$.} The associated projector is:
\[\begin{array}{rcl}
P_{\xplus\xplus} &=& (P_{\xplus}\otimes P_{\xplus}) + (P_{\xminus} \otimes P_{\xminus})\\
\\
&=& \op{0}{0} \otimes \op{0}{0} + \op{1}{1} \otimes \op{1}{1} \\
\\
&=& \op{00}{00} + \op{11}{11} 
\end{array}\]
The associated probability is 
$\ip{\phi}{P_{\xplus\xplus}\ket{\phi}} = \frac{1}{\sqrt{2}} \bra{\phi}(\ket{00}+\ket{11}) 
= \frac{1}{2} (\ip{00}{00} + \ip{11}{11}) = 1$.

\paragraph*{Experiment $\yplus\yplus$.} When calculating the
associated operator we only focus on the projectors that will actually
interact with $\ket{\phi}$: 
\[\begin{array}{rcl}
P_{\yplus\yplus} &=& (P_{\yplus}\otimes P_{\yplus}) + (P_{\yminus} \otimes P_{\yminus})\\
\\
&=& \ldots 
+   \frac{1}{16} \op{00}{00} + \frac{3}{16}\op{00}{11} + \frac{3}{16}\op{11}{00} + \frac{9}{16} \op{11}{11}\\
\\
&& + 
\frac{9}{16} \op{00}{00} + \frac{3}{16}\op{00}{11} + \frac{3}{16}\op{11}{00} + \frac{1}{16} \op{11}{11} \\
\\
&=& \ldots + \frac{10}{16} \op{00}{00} + \frac{6}{16}\op{00}{11} + \frac{6}{16}\op{11}{00} + \frac{10}{16} \op{11}{11} \\
\end{array}\]
The associated probability is:
\[
  \frac{1}{2} \bra{\phi} (\frac{16}{16} \ket{00} + \frac{16}{16} \ket{11}) = 
  \srtt \bra{\phi} (\ket{00} + \ket{11}) =  1
\]

\newpage
%% we only care about the parts that have the same values in a key
%% like |aa><bb| over a,b in {0,1}
\paragraph*{Experiment $\zplus\zplus$.}
We show some work in calculating. First:
\[
    \begin{array}{rcl}
    P_{\zplus}\otimes P_{\zplus}
        &=& (\frac{1}{4} \op{0}{0} 
         + \frac{\sqrt{3}}{4} \op{0}{1} + \frac{\sqrt{3}}{4} 
             \op{1}{0} + \frac{3}{4} \op{1}{1}) \otimes (\frac{1}{4} \op{0}{0} 
        + \frac{\sqrt{3}}{4} \op{0}{1} + \frac{\sqrt{3}}{4} \op{1}{0} 
        + \frac{3}{4} \op{1}{1})
        \\ \\
        &=& \dots + \frac{1}{16}\op{00}{00}
          + \frac{3}{16}\op{00}{11}
          + \frac{3}{16}\op{11}{00}
          + \frac{9}{16}\op{11}{11}
    \end{array}
\]

\[
    \begin{array}{rcl}
    P_{\zminus}\otimes P_{\zminus}
        &=& (\frac{3}{4} \op{0}{0} 
            - \frac{\sqrt{3}}{4} \op{0}{1} 
            - \frac{\sqrt{3}}{4} \op{1}{0} 
            + \frac{1}{4} \op{1}{1})
            \otimes 
          (\frac{3}{4} \op{0}{0} 
            - \frac{\sqrt{3}}{4} \op{0}{1} 
            - \frac{\sqrt{3}}{4} \op{1}{0} 
            + \frac{1}{4} \op{1}{1})
            \\ \\
        &=& \dots 
        + \frac{9}{16} \op{00}{00}
        + \frac{3}{16} \op{00}{11} 
        + \frac{3}{16} \op{11}{00} 
        + \frac{1}{16} \op{11}{11} 
    \end{array}
\]

The projector:
\[
    \begin{array}{rcl}
P_{\zplus\zplus} 
    &=& (P_{\zplus}\otimes P_{\zplus}) + (P_{\zminus} \otimes P_{\zminus})\\ \\
    &=& \dots 
        + \frac{10}{16} \op{00}{00}
        + \frac{6}{16} \op{00}{11} 
        + \frac{6}{16} \op{11}{00} 
        + \frac{10}{16} \op{11}{11} 
\end{array}\]
Now we apply $P_{\zplus\zplus}$ to $\ket{\phi}$:

\[\begin{array}{rcl}
P_{\zplus\zplus}\ket{\phi} 
    &=& (\frac{10}{16} \op{00}{00}
        + \frac{6}{16} \op{00}{11} 
        + \frac{6}{16} \op{11}{00} 
        + \frac{10}{16} \op{11}{11} )(\frac{1}{\sqrt{2}}(\ket{00} + \ket{11}))
    \\ \\
    &=& \srtt(\frac{10}{16}\ket{00}\ip{00}{00}
        + \frac{10}{16}\ket{00}\ip{00}{11}
        + \frac{6}{16}\ket{00}\ip{11}{00}
        + \frac{6}{16}\ket{00}\ip{11}{11}
        \\ \\
        && +  \frac{6}{16}\ket{11}\ip{00}{00}
        + \frac{6}{16}\ket{11}\ip{00}{11}
        + \frac{10}{16} \ket{11}\ip{11}{00}
        + \frac{10}{16} \ket{11}\ip{11}{11}
        )
    \\ \\
    &=& \srtt(\frac{10}{16}\ket{00}\ip{00}{00}
        + \frac{6}{16}\ket{00}\ip{11}{11} 
        +  \frac{6}{16}\ket{11}\ip{00}{00}
        + \frac{10}{16} \ket{11}\ip{11}{11}
        )
    \\ \\
    &=& \srtt(\frac{10}{16}\ket{00}
        + \frac{6}{16}\ket{00}
        +  \frac{6}{16}\ket{11}
        + \frac{10}{16} \ket{11}
        )
    \\ \\
    &=& \srtt(\frac{16}{16}\ket{00} + \frac{16}{16} \ket{11})
    \\ \\
    &=& \srtt(\ket{00} + \ket{11})
\end{array}\]
\\
Next we calculate 
$\ip{\phi}{P_{\zplus\zplus}\ket{\phi}} = \ip{\phi}{\srtt(\ket{00} + \ket{11})}$
\[
    \begin{array}{rcl}
    \ip{\phi}{\srtt(\ket{00} + \ket{11})}
        &=& \ip{\srtt(\ket{00} + \ket{11})}{\srtt(\ket{00} + \ket{11})} 
        \\ \\
        &=& \frac{1}{2}(\ip{00}{00} + \ip{11}{11})
        \\ \\
        &=& \frac{1}{2}(1 + 1) 
        \\ \\
        &=& 1
    \end{array}
\]
Therefore the associated probability for 
$\ip{\phi}{P_{\zplus\zplus}\ket{\phi}}$ is $1$.

\newpage
%%% X+ Y+
\paragraph*{Experiment $\xplus\yplus$.} The associated projector is:
\[\begin{array}{rcl}
P_{\xplus\yplus}
&=& (P_{\xplus}\otimes P_{\yplus}) + (P_{\xminus} \otimes P_{\yminus})
    \\ \\
&=& \frac{1}{4} \op{00}{00} + \frac{1}{4} \op{11}{11} 
\end{array}\]
The associated probability is:
\[
  \frac{1}{2} \bra{\phi} (\frac{1}{4} \ket{00} + \frac{1}{4} \ket{11}) = 
  \frac{1}{2} (\frac{1}{4} + \frac{1}{4}) = \frac{1}{4}
\]

%%% begin experiment X+ Z+
\paragraph*{Experiment $\xplus\zplus$.} 
We show some work in calculating. First:

\[
    \begin{array}{rcl}
    P_{\xplus}\otimes P_{\zplus}
        &=& \op{0}{0}  
             \otimes 
            (\frac{1}{4} \op{0}{0} 
                + \frac{\sqrt{3}}{4} \op{0}{1} 
                + \frac{\sqrt{3}}{4} \op{1}{0} 
                + \frac{3}{4} \op{1}{1})
        \\ \\
        &=& \dots + \frac{1}{4}\op{00}{00} 
    \end{array}
\]

\[
    \begin{array}{rcl}
    P_{\xminus}\otimes P_{\zminus}
        &=& \op{1}{1}
            \otimes 
          (\frac{3}{4} \op{0}{0} 
            - \frac{\sqrt{3}}{4} \op{0}{1} 
            - \frac{\sqrt{3}}{4} \op{1}{0} 
            + \frac{1}{4} \op{1}{1})
            \\ \\
        &=& \dots + \frac{1}{4}\op{11}{11} 
    \end{array}
\]


The projector:
\[
    \begin{array}{rcl}
        P_{\xplus\zplus} 
        &=& (P_{\xplus}\otimes P_{\zplus}) + (P_{\xminus} \otimes P_{\zminus}) 
        \\ \\
        &=& \frac{1}{4}\op{00}{00} + \frac{1}{4}\op{11}{11}
    \end{array}
\]

Now we apply $P_{\xplus\zplus}$ to $\ket{\phi}$:

\[
    \begin{array}{rcl}
        P_{\xplus\zplus}\ket{\phi} &=& 
        (\frac{1}{4}\op{00}{00} + \frac{1}{4}\op{11}{11})(\srtt(\ket{00} + \ket{11}))
        \\ \\
        &=&
        \srtt(\frac{1}{4}\ket{00}\ip{00}{00} 
        + \frac{1}{4}\ket{00}\ip{00}{11} 
        + \frac{1}{4}\ket{11}\ip{11}{00}
        + \frac{1}{4}\ket{11}\ip{11}{11})
        \\ \\
        &=& 
        \srtt(\frac{1}{4}\ket{00}\ip{00}{00} + \frac{1}{4}\ket{11}\ip{11}{11})
        \\ \\
        &=& 
        \srtt(\frac{1}{4}\ket{00} + \frac{1}{4}\ket{11})
        \\ \\
    \end{array}
\]

Next we calculate 
$\ip{\phi}{P_{\xplus\zplus}\ket{\phi}} = \ip{\phi}{ \srtt(\frac{1}{4}\ket{00} + \frac{1}{4}\ket{11})}$
\[
    \begin{array}{rcl}
        \ip{\phi}{\srtt(\frac{1}{4}\ket{00} + \frac{1}{4}\ket{11})} 
        &=& 
        \ip{\srtt(\ket{00} + \ket{11})}{\srtt(\frac{1}{4}\ket{00} + \frac{1}{4}\ket{11})} 
        \\ \\
        &=& \frac{1}{2}\ip{\ket{00} + \ket{11}}{\frac{1}{4}(\ket{00} + \ket{11})}
        \\ \\
        &=& \frac{1}{2}(\frac{1}{4}\ip{\ket{00} + \ket{11}}{00} 
        + \frac{1}{4}\ip{\ket{00} + \ket{11}}{11})
        \\ \\
        &=& \frac{1}{8}(\ip{00}{00} + \ip{11}{00} + \ip{00}{11} + \ip{11}{11} )
        \\ \\
        &=& \frac{1}{8}(\ip{00}{00} + \ip{11}{11})
        \\ \\
        &=& \frac{1}{4}
    \end{array}
\]

%%% end of experiment x+ z+

%%% begin experiment y+ x+
\paragraph*{Experiment $\yplus\xplus$.} 
We show some work in calculating. First:

\[
    \begin{array}{rcl}
    P_{\yplus}\otimes P_{\xplus}
        &=& 
        (\frac{1}{4} \op{0}{0} 
        - \frac{\sqrt{3}}{4} \op{0}{1} 
        - \frac{\sqrt{3}}{4} \op{1}{0} 
        + \frac{3}{4} \op{1}{1})
      \otimes 
        \op{0}{0}  
        \\ \\
        &=&  \frac{1}{4} \op{00}{00} 
    \end{array}
\]

\[
    \begin{array}{rcl}
    P_{\yminus}\otimes P_{\xminus}
        &=& (\frac{3}{4} \op{0}{0} 
            + \frac{\sqrt{3}}{4} \op{0}{1} 
            + \frac{\sqrt{3}}{4} \op{1}{0} 
            + \frac{1}{4} \op{1}{1})
        \otimes \op{1}{1}
            \\ \\
        &=&  \frac{1}{4} \op{11}{11} 
    \end{array}
\]


The projector:
\[
    \begin{array}{rcl}
        P_{\yplus\xplus} 
        &=& (P_{\yplus}\otimes P_{\xplus}) + (P_{\yminus} \otimes P_{\xminus}) 
        \\ \\
        &=&  \frac{1}{4} \op{00}{00} + \frac{1}{4} \op{11}{11} 
    \end{array}
\]

Now we apply $P_{\yplus\xplus}$ to $\ket{\phi}$:

\[
    \begin{array}{rcl}
        P_{\yplus\xplus}\ket{\phi} 
        &=&  (\frac{1}{4} \op{00}{00} + \frac{1}{4} \op{11}{11})
        (\srtt(\ket{00} + \ket{11}))
        \\ \\
        &=& \srtt(\frac{1}{4}(\ket{00} + \ket{11}))
    \end{array}
\]

Next we calculate 
$\ip{\phi}{P_{\yplus\xplus}\ket{\phi}} = \ip{\phi}{\srtt(\frac{1}{4}(\ket{00} + \ket{11}))}$
\[
    \begin{array}{rcl}
        \ip{\phi}{\srtt(\frac{1}{4}(\ket{00} + \ket{11}))}
        &=& \frac{1}{2}\ip{\ket{00} + \ket{11}}{\frac{1}{4}(\ket{00} + \ket{11})}
        \\ \\ 
        &=& \frac{1}{8}(\ip{00}{\ket{00} + \ket{11}} + \ip{11}{\ket{00} + \ket{11}})
        \\ \\
        &=& \frac{1}{8}(\ip{00}{00} + \ip{00}{11} + \ip{11}{00} + \ip{11}{11})
        \\ \\
        &=& \frac{1}{8}(1 + 1) 
        \\ \\
        &=& \frac{1}{4}
    \end{array}
\]

%%% end of experiment y+ x+


%%% begin experiment y+ Z+
\paragraph*{Experiment $\yplus\zplus$.} 
We show some work in calculating. First:

\[
    \begin{array}{rcl}
    P_{\yplus}\otimes P_{\zplus}
        &=& (\frac{1}{4} \op{0}{0} 
                - \frac{\sqrt{3}}{4} \op{0}{1} 
                - \frac{\sqrt{3}}{4} \op{1}{0} 
                + \frac{3}{4} \op{1}{1})
             \otimes 
            (\frac{1}{4} \op{0}{0} 
                + \frac{\sqrt{3}}{4} \op{0}{1} 
                + \frac{\sqrt{3}}{4} \op{1}{0} 
                + \frac{3}{4} \op{1}{1})
        \\ \\
        &=& \dots 
        + \frac{1}{16}\op{00}{00}
        - \frac{3}{16}\op{00}{11}
        - \frac{3}{16}\op{11}{00}
        + \frac{9}{16}\op{11}{11}
    \end{array}
\]

\[
    \begin{array}{rcl}
    P_{\yminus}\otimes P_{\zminus}
        &=& (\frac{3}{4} \op{0}{0} 
            + \frac{\sqrt{3}}{4} \op{0}{1} 
            + \frac{\sqrt{3}}{4} \op{1}{0} 
            + \frac{1}{4} \op{1}{1})
            \otimes 
          (\frac{3}{4} \op{0}{0} 
            - \frac{\sqrt{3}}{4} \op{0}{1} 
            - \frac{\sqrt{3}}{4} \op{1}{0} 
            + \frac{1}{4} \op{1}{1})
            \\ \\
        &=& \dots 
        + \frac{9}{16}\op{00}{00}
        - \frac{3}{16}\op{00}{11}
        - \frac{3}{16}\op{11}{00}
        + \frac{1}{16}\op{11}{11} 
    \end{array}
\]


The projector:
\[
    \begin{array}{rcl}
        P_{\yplus\zplus} 
        &=& (P_{\yplus}\otimes P_{\zplus}) + (P_{\yminus} \otimes P_{\zminus}) 
        \\ \\
        &=& \frac{1}{16}\op{00}{00}
        - \frac{3}{16}\op{00}{11}
        - \frac{3}{16}\op{11}{00}
        + \frac{9}{16}\op{11}{11} 
        \\ \\
        &&
        + \frac{9}{16}\op{00}{00}
        - \frac{3}{16}\op{00}{11}
        - \frac{3}{16}\op{11}{00}
        + \frac{1}{16}\op{11}{11} 
        \\ \\
        &=& 
          \frac{10}{16}\op{00}{00}
        - \frac{6}{16}\op{00}{11}
        - \frac{6}{16}\op{11}{00}
        + \frac{10}{16}\op{11}{11} 
    \end{array}
\]

Now we apply $P_{\yplus\zplus}$ to $\ket{\phi}$:

\[
    \begin{array}{rcl}
        P_{\yplus\zplus}\ket{\phi} &=& 
        (\frac{10}{16}\op{00}{00}
        - \frac{6}{16}\op{00}{11}
        - \frac{6}{16}\op{11}{00}
        + \frac{10}{16}\op{11}{11})(\srtt(\ket{00} + \ket{11}))
        \\ \\
        &=& \srtt(\frac{10}{16}\ket{00}\ip{00}{00}
        + \frac{10}{16}\ket{00}\ip{00}{11}
        - \frac{6}{16}\ket{00}\ip{11}{00}
        - \frac{6}{16}\ket{00}\ip{11}{11}
        \\ \\ 
        &&
        - \frac{6}{16}\ket{11}\ip{00}{00}
        - \frac{6}{16}\ket{11}\ip{00}{11}
        + \frac{10}{16}\ket{11}\ip{11}{00}
        + \frac{10}{16}\ket{11}\ip{11}{11}
        )
        \\ \\
        &=&
        \srtt(\frac{10}{16}\ket{00}\ip{00}{00}
        - \frac{6}{16}\ket{00}\ip{11}{11}
        - \frac{6}{16}\ket{11}\ip{00}{00}
        + \frac{10}{16}\ket{11}\ip{11}{11}
        )
        \\ \\
        &=&
        \srtt(\frac{4}{16}\ket{00} + \frac{4}{16}\ket{11})
        \\ \\
        &=&
        \srtt(\frac{1}{4}\ket{00} + \frac{1}{4}\ket{11})
    \end{array}
\]

Next we calculate 
$\ip{\phi}{P_{\yplus\zplus}\ket{\phi}} = \ip{\phi}{\srtt(\frac{1}{4}\ket{00} + \frac{1}{4}\ket{11})}$
\[
    \begin{array}{rcl}
        \ip{\phi}{\srtt(\frac{1}{4}\ket{00} + \frac{1}{4}\ket{11})}
        &=& \frac{1}{2}(\ip{\ket{00} + \ket{11}}{\frac{1}{4}(\ket{00} + \ket{11})})
        \\ \\
        &=& \frac{1}{8}(\ip{00}{00} + \ip{11}{11})
        \\  \\
        &=&
        \frac{1}{4}
    \end{array}
\]

%%% end of experiment y+ z+


%%% begin experiment z+ x+
\paragraph*{Experiment $\zplus\xplus$.} 
We show some work in calculating. First:

\[
    \begin{array}{rcl}
    P_{\zplus}\otimes P_{\xplus}
        &=& 
            (\frac{1}{4} \op{0}{0} 
                + \frac{\sqrt{3}}{4} \op{0}{1} 
                + \frac{\sqrt{3}}{4} \op{1}{0} 
                + \frac{3}{4} \op{1}{1})
        \otimes \op{0}{0}  
        \\ \\
        &=& \dots + \frac{1}{4}\op{00}{00} 
    \end{array}
\]

\[
    \begin{array}{rcl}
    P_{\zminus}\otimes P_{\xminus}
        &=& (\frac{3}{4} \op{0}{0} 
            - \frac{\sqrt{3}}{4} \op{0}{1} 
            - \frac{\sqrt{3}}{4} \op{1}{0} 
            + \frac{1}{4} \op{1}{1})
        \otimes\op{1}{1}
            \\ \\
        &=& \dots + \frac{1}{4}\op{11}{11} 
    \end{array}
\]


The projector:
\[
    \begin{array}{rcl}
        P_{\zplus\xplus} 
        &=& (P_{\zplus}\otimes P_{\xplus}) + (P_{\zminus} \otimes P_{\xminus}) 
        \\ \\
        &=& \frac{1}{4}\op{00}{00} + \frac{1}{4}\op{11}{11}
    \end{array}
\]

Now we apply $P_{\zplus\xplus}$ to $\ket{\phi}$:

\[
    \begin{array}{rcl}
        P_{\zplus\xplus}\ket{\phi} &=& 
        (\frac{1}{4}\op{00}{00} + \frac{1}{4}\op{11}{11})(\srtt(\ket{00} + \ket{11}))
        \\ \\
        &=&
        \srtt(\frac{1}{4}(\ket{00}\ip{00}{00} 
        + \ket{00}\ip{00}{11} 
        + \ket{11}\ip{11}{00}
        + \ket{11}\ip{11}{11}))
        \\ \\
        &=& 
        \srtt(\frac{1}{4}(\ket{00}\ip{00}{00} + \ket{11}\ip{11}{11}))
        \\ \\
        &=& 
        \srtt(\frac{1}{4}\ket{00} + \frac{1}{4}\ket{11})
        \\ \\
    \end{array}
\]

Next we calculate 
$\ip{\phi}{P_{\zplus\xplus}\ket{\phi}} 
= \ip{\phi}{\srtt(\frac{1}{4}\ket{00} + \frac{1}{4}\ket{11})}$
\[
    \begin{array}{rcl}
        \ip{\phi}{\srtt(\frac{1}{4}\ket{00} + \frac{1}{4}\ket{11})} 
        &=& 
        \ip{\srtt(\ket{00} + \ket{11})}{\srtt(\frac{1}{4}\ket{00} + \frac{1}{4}\ket{11})} 
        \\ \\
        &=& \frac{1}{2}\ip{\ket{00} + \ket{11}}{\frac{1}{4}(\ket{00} + \ket{11})}
        \\ \\
        &=& \frac{1}{2}(\frac{1}{4}\ip{\ket{00} + \ket{11}}{00} 
        + \frac{1}{4}\ip{\ket{00} + \ket{11}}{11})
        \\ \\
        &=& \frac{1}{8}(\ip{00}{00} + \ip{11}{00} + \ip{00}{11} + \ip{11}{11} )
        \\ \\
        &=& \frac{1}{8}(\ip{00}{00} + \ip{11}{11})
        \\ \\
        &=& \frac{1}{4}
    \end{array}
\]

%%% end of experiment z+ x+



%%% begin experiment z+ y+
\paragraph*{Experiment $\zplus\yplus$.} 
We show some work in calculating. First:

\[
    \begin{array}{rcl}
    P_{\zplus}\otimes P_{\yplus}
        &=& 
            (\frac{1}{4} \op{0}{0} 
                + \frac{\sqrt{3}}{4} \op{0}{1} 
                + \frac{\sqrt{3}}{4} \op{1}{0} 
                + \frac{3}{4} \op{1}{1})
        \otimes (\frac{1}{4} \op{0}{0} 
                - \frac{\sqrt{3}}{4} \op{0}{1} 
                - \frac{\sqrt{3}}{4} \op{1}{0} 
                + \frac{3}{4} \op{1}{1})
        \\ \\
        &=& \dots 
        + \frac{1}{16} \op{00}{00}
        - \frac{3}{16} \op{00}{11}
        - \frac{3}{16} \op{11}{00}
        + \frac{9}{16} \op{11}{11}
    \end{array}
\]

\[\begin{array}{rcl}
    P_{\zminus}\otimes P_{\yminus}
        &=& (\frac{3}{4} \op{0}{0} 
            - \frac{\sqrt{3}}{4} \op{0}{1} 
            - \frac{\sqrt{3}}{4} \op{1}{0} 
            + \frac{1}{4} \op{1}{1})
        \otimes (\frac{3}{4} \op{0}{0} 
            + \frac{\sqrt{3}}{4} \op{0}{1} 
            + \frac{\sqrt{3}}{4} \op{1}{0} 
            + \frac{1}{4} \op{1}{1})
            \\ \\
        &=& \dots 
        + \frac{9}{16} \op{00}{00}
        - \frac{3}{16} \op{00}{11}
        - \frac{3}{16} \op{11}{00}
        + \frac{1}{16} \op{11}{11}
    \end{array}
\]

The projector:
\[\begin{array}{rcl}
P_{\zplus\yplus} 
        &=& (P_{\zplus}\otimes P_{\yplus}) + (P_{\zminus} \otimes P_{\yminus}) 
        \\ \\
        &=& \dots 
        + \frac{10}{16} \op{00}{00}
        - \frac{6}{16} \op{00}{11}
        - \frac{6}{16} \op{11}{00}
        + \frac{10}{16} \op{11}{11}
    \end{array}
\]

Now we apply $P_{\zplus\yplus}$ to $\ket{\phi}$:

\[
    \begin{array}{rcl}
        P_{\zplus\yplus}\ket{\phi} 
        &=& 
        (\frac{10}{16} \op{00}{00}
        - \frac{6}{16} \op{00}{11}
        - \frac{6}{16} \op{11}{00}
        + \frac{10}{16} \op{11}{11})(\srtt(\ket{00} + \ket{11}))
        \\ \\
        &=& \srtt(
        \frac{10}{16} \ket{00}\ip{00}{00}
        + \frac{10}{16} \ket{00}\ip{00}{11}
        - \frac{6}{16} \ket{00}\ip{11}{00}
        - \frac{6}{16} \ket{00}\ip{11}{11}
        \\ \\
        && 
        - \frac{6}{16} \ket{11}\ip{00}{00}
        - \frac{6}{16} \ket{11}\ip{00}{11}
        + \frac{10}{16} \ket{11}\ip{11}{00}
        + \frac{10}{16} \ket{11}\ip{11}{11}
        )
        \\ \\
        &=& \srtt(
        \frac{10}{16} \ket{00}\ip{00}{00}
        - \frac{6}{16} \ket{00}\ip{11}{11}
        - \frac{6}{16} \ket{11}\ip{00}{00}
        + \frac{10}{16} \ket{11}\ip{11}{11}
        )
        \\ \\
        &=& \srtt(
        \frac{10}{16} \ket{00}
        - \frac{6}{16} \ket{00}
        - \frac{6}{16} \ket{11}
        + \frac{10}{16} \ket{11}
        )
        \\ \\
        &=& \srtt( \frac{4}{16} \ket{00} + \frac{4}{16} \ket{11})
        \\ \\
        &=& \srtt( \frac{1}{4} \ket{00} + \frac{1}{4} \ket{11})
    \end{array}
\]

Next we calculate 
$\ip{\phi}{P_{\zplus\yplus}\ket{\phi}} = \ip{\phi}{\srtt( \frac{1}{4} \ket{00} + \frac{1}{4} \ket{11})}$
\[
    \begin{array}{rcl}
        \srtt\ip{\srtt(\ket{00} + \ket{11})}{(\frac{1}{4} \ket{00} + \frac{1}{4} \ket{11})}
        &=& \frac{1}{2}(\frac{1}{4}\ip{00}{00} + \frac{1}{4}\ip{11}{11})
        \\ \\
        &=& \frac{1}{2}(\frac{1}{2})
        \\ \\
        &=& \frac{1}{4}
    \end{array}
\]

%%% end of experiment z+ y+

\section{Insights}

Now that you have seen the precise mathematical description of
  the experiment, do you have any new insights regarding the \emph{mystery}?

It is hard to say now since the mystery has been explained as a different way of viewing
the problem. I think the new insight is that there is not tangible connectedness,
but the context of each question is built on the projectors and their orientations,
which have this interesting property in the probabilities both say yes/no. 
My best explanation would be that by setting the devices to the same setting,
you are aligning the projection for the partciles, so when you ask yes or no, their
answers are aligned. With different projectors, there's 4 possibilities of yes/yes, yes/no, no/yes, no/no -- then the alignment cannot help really, so a quarter of the time they
line up (yes/yes). 

\section{Qiskit}
Implement the 9 experiments in Qiskit. Experiment-$\xplus\xplus$
  uses the default projector so it is straightforward to realize in
  Qiskit using the build-in measurement operator. We illustrate how to
  realize a measurement using another projector below.


  First our projector operators are all of the form
  $\op{R\ket{0}}{R\ket{0}}$ for some unitary operator $R$. It is
  important to recall that for unitary operators $R$, we have
  $R^\dagger = R^{-1}$ and $\ip{a}{b} = \ip{Ra}{Rb}$. Now our desired
  probabilities involve computations
  $\ip{\phi}{R\ket{0}}\ip{R\ket{0}}{\phi}$ and these can be simplified
  as follows:
  \[\begin{array}{rcl}
      \ip{\phi}{R\ket{0}}\ip{R\ket{0}}{\phi} &=&
        \ip{R^\dagger\phi}{R^\dagger R\ket{0}}\ip{R^\dagger R\ket{0}}{R^\dagger\phi} \\
\\
&=& \ip{R^\dagger\phi}{0}\ip{0}{R^\dagger\phi} 
 \end{array}\]

In other words, instead of using projector $\op{R\ket{0}}{R\ket{0}}$ on
state $\ket{\phi}$, it is equivalent to use projector $\op{0}{0}$ on
state $R^\dagger\ket{\phi}$. 



%%%%%%%%%%%%%%%%%%%%%%%%%%%%%%%%%%%%%%%%%%%%%%%%%%%%%%%%%%%%%%%%%%%%%%%%

\end{document}


